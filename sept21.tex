\subsection{September 21 Group Assignment} Let $X=[0,1)$ and $m^{*}$ denote Lebesgue outer measure.
\begin{enumerate}
\item Prove that $A\subseteq \R$ and $\lambda \in \R$ then $m^{*}(A)=m^{*}(A+\lambda)$. \\
\textbf{Lemma:} For any set $A$, $A \subseteq \bigcup_{i=1}^\infty (a_i, b_i)$ if and only if $A + \lambda \subseteq \bigcup_{i=1}^\infty (a_i + \lambda, b_i + \lambda)$. 
\begin{pf}
Assume $A \subseteq \bigcup_{i=1}^\infty (a_i, b_i)$. Then, let $a \in A + \lambda$. Then, $a - \lambda \in A$ implies there exists some $i$ such that $a - \lambda \in (a_i, b_i)$. Equivalently, $a_i < a - \lambda < b_i$ and $a_i + \lambda < a < b_i + \lambda$. Thus, $a \in (a_i + \lambda, b_i + \lambda)$ for some $i$ which implies $A + \lambda \subseteq \bigcup_{i=1}^\infty (a_i + \lambda, b_i + \lambda)$.\\
\\
Assume $A + \lambda \subseteq \bigcup_{i=1}^\infty (a_i + \lambda, b_i + \lambda)$. Suppose $a \in A$. Then, $a+\lambda \in A + \lambda$ which implies there exists $i$ such that $a + \lambda \in (a_i + \lambda, b_i + \lambda)$. Thus $a_i+\lambda \leq a + \lambda \leq b_i + \lambda$ and so $a_i \leq a \leq b_i$. Thus, $A \subseteq \bigcup_{i=1}^\infty (a_i, b_i)$. 
\end{pf}
Prove that $A\subseteq \R$ and $\lambda \in \R$ then $m^{*}(A)=m^{*}(A+\lambda)$.
\begin{pf}
	Notice $m^{*}(A)= \inf\left\lbrace \sum_{i=1}^\infty (b_i - a_i) \colon \ A \subseteq \bigcup_{i=1}^\infty (a_i, b_i) \right\rbrace$. From the previous lemma we know $A \subseteq \bigcup_{i=1}^\infty (a_i, b_i) $ implies $A + \lambda \subseteq \bigcup_{i=1}^\infty (a_i + \lambda, b_i + \lambda) $. Thus, we can write $m^{*}(A+ \lambda)= \inf\left\lbrace \sum_{i=1}^\infty (b_i +\lambda - (a_i + \lambda)) \colon \ A + \lambda \subseteq \bigcup_{i=1}^\infty (a_i + \lambda, b_i + \lambda) \right\rbrace$. Since $\sum_{i=1}^\infty (b_i +\lambda - (a_i + \lambda)) = \sum_{i=1}^\infty (b_i - a_i)$, $m^{*}(A)=m^{*}(A+\lambda)$.
\end{pf}


\item Prove that if $m^{*}(A)=0$, then $A$ is $m^{*}$-measurable.
	\begin{pf}
Assume $m^{*}(A)=0$. Then, for any set $B \subset \R$, $B \cap A \subseteq A$ and $B \cap A^c \subseteq B$. By monotonicity of $m^{*}$, $m^{*}(B \cap A) \leq m^{*}(A)$ and $m^{*}(B \cap A^c) \leq m^{*}(B)$. Thus, 
\[
m^{*}(B \cap A) + m^{*}(B \cap A^c) \leq m^{*}(A) +  m^{*}(B)= 0 + m^{*}(B);
\]
\[
\text{ equivalently,  } m^{*}(B \cap A) + m^{*}(B \cap A^c) \leq m^{*}(B).
\]
Notice  $B=(B \cap A) \cup (B \cap A^c)$. Additionally, by subadditivity of $m^{*}$,
\[ m^{*}(B)=m^{*}((B \cap A) \cup (B \cap A^c)) \leq m^{*}(B \cap A) + m^{*}(B \cap A^c) \text{ and so } m^{*}(B)\leq m^{*}(B \cap A) + m^{*}(B \cap A^c).
\]
Hence for any $B \subset \R$, $m^{*}(B) = m^{*}(B \cap A) + m^{*}(B \cap A^c)$ which implies $A$ is $m^{*}$-measurable.
\end{pf}

\item Prove that if $E$ is an $m^{*}$-measurable subset of $X$ and $E \subseteq N$, where $N$ is the non-measurable set in section 1.1, then $m^{*}(E)=0$. 
\begin{pf}
Assume $E$ is an $m^{*}$-measurable subset of $X$ and $E \subseteq N$, where $N$ is described below.
\textbf{(construction of $N$ from $\S 1.1$)} Define an equivalence relation on $[0,1)$ by declaring that $x \sim y$ if and only if $x - y \in \Q$. By the axiom of choice, we can let $N$ be a subset of $[0,1)$ that contains exactly one member of each equivalence class. Consider $R=\Q \cap [0,1)$ and for each $r \in R$, let
$
N_r=\left\lbrace x + r: x \in N \cap [0, 1-r) \right\rbrace \cup \left\lbrace x+r-1 : x \in N\cap [1-r, 1) \right\rbrace
$
We showed in class that $N_r$ is not measurable. By exercise 1 from September 21, $m^{*}$ is translation invariant so $N$ is not measurable.
Next, consider $E \subseteq N$. Then, $E$ contains one element or none from each of the equivalence classes defined above. 
Following the construction of $N_r$, let  $E_r=\left\lbrace x + r: x \in E \cap [0, 1-r) \right\rbrace \cup \left\lbrace x+r-1 : x \in E\cap [1-r, 1) \right\rbrace.$
Thus, for every $r \in R$, $E_r \subseteq N_r$. $E_r$ is just $E$ translated $r$ units and then translated back into $[0,1)$ as needed, so from exercise 1, $m^{*}(E)=m^{*}(E_r)$. For any $r \in R$, $E_r \subset [0,1)$ so $\bigcup_{r \in R} E_r \subset [0, 1)$. In class we showed for all $r, s \in R, r\neq s$ $N_r \cap N_s = \o$, thus $\bigcup_{r \in R} E_r$ is a disjoint union. Therefore, $m^{*}(\bigcup_{r \in R} E_r) = \sum_{r \in R}m^{*}(E_r)$. Since $m^{*}(E_r)=m^{*}(E)$, $m^{*}(\bigcup_{r \in R} E_r) = \sum_{r \in R}m^{*}(E) = m^{*}(E)\sum_{r \in R}(1)=m^{*}(E) \cdot \infty$. By monotonicity of $m^{*}$, $m^{*}(\bigcup_{r \in R} E_r)\leq m^{*}([0,1))=1$.   Thus $m^{*}(\bigcup_{r \in R} E_r)\leq 1$ and $m^{*}(\bigcup_{r \in R} E_r)=m^{*}(E) \cdot \infty$. If $m^{*}(E)>0$, the previous equality would imply $m^{*}(\bigcup_{r \in R} E_r)= \infty$ which is false since $m^{*}(\bigcup_{r \in R} E_r)\leq 1$; thus $m^{*}(E)=0$.
\end{pf}
\item If $E$ is measurable and $m^{*}(E)>0$, then $E$ contains a non-measurable subset. (this is exercise 29 on pg 39 in Rolland)
\begin{pf}
Assume $E$ is measurable with $m^*(E) > 0$. Without loss of generality, assume $E \subset [0,1)$. Assume that all subsets of $E$ are measurable. Then, using the construction of $N_r$ from $\S 1.1$, $E = \cup_{r \in R}(E \cap N_r)$. Then, from (3), $m^*(E \cap N_r)=0$ implies $E \cap N_r \subset E$ are measurable for all $r \in R$.  Since  $E = \cup_{r \in R}(E \cap N_r)$, $m^*(E)=m^*(\cup_{r \in R}(E \cap N_r))$. Since $E \cap N_{r_1}$ and $E \cap N_{r_2}$ are disjoint for any $r_1, r_2 \in R$, $m^*(\cup_{r \in R}(E \cap N_r))= \sum_{r\in R}m^*(E \cap N_r)=\sum_{r \in R}0=0$. Thus, $m^*(E)=0$. But, $m^*(E)>0$ so there must be some $r$ for which $E \cap N_r$ is not measurable. Thus, $E \cap N_r \subseteq E$, so $E$ contains a non-measurable subset.  \\
\url{http://math.ucr.edu/~edwardb/Graduate%20Classes/Math%20209A/Homework%201.pdf}
\url {http://www.math.ttu.edu/~drager/Classes/01Fall/reals/ans2.pdf}
\end{pf}

\end{enumerate}