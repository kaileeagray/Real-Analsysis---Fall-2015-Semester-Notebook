\begin{enumerate}
\item Let $X=Y=[0,1]$, $\M = \N = B_{[0,1]}$	, $\mu$ is Lebesgue measure, and $\nu$ is the counting measure. If $D=\{(x,x): x \in [0,1]\}$ then $\int \int \chi_D d \mu d\nu$, $\int \int \chi_D d \nu d \mu$, and $\int \chi_D d (\mu \times \nu)$ are all unequal. Verify this and explain why this does not contradict the Fubini-Tonelli Theorem.
\begin{pf}
First, calculate $\int \int \chi_D d \mu d\nu$, $\int \int \chi_D d \nu d \mu$: 
\begin{eqnarray*}
 	\int \int \chi_D d \mu d\nu & = & \int \int \left(\chi_D \right)^y d \mu d \nu \\
 	& = & \int \mu(D^y) d \nu \\
 	& = & \int \mu ( \{ x \}) d \nu \\
 	& = & \int 0 d \nu \\
 	& = & 0\\
 	\int \int \chi_D d \nu d\mu & = & \int \int \left(\chi_D \right)_x  d \nu d \mu \\
 	& = & \int \nu(D^y) d \mu \\
 	& = & \int \nu ( \{ x \}) d \mu \\
 	& = & \int 1 d \mu \\
 	& = & 1\cdot\mu([0,1])\\
 	& = & 1.
 \end{eqnarray*}
 Next, consider $\int \chi_D d (\mu \times \nu)=(\mu \times \nu)(D)$.  By definition, \[
 (\mu \times \nu)(D)=\inf \left\{ (\mu \times \nu)\left( \bigcup_1^\infty (A_i \times B_i)\right): D \subseteq \bigcup_1^\infty (A_i \times B_i) \right\}.
 \]
Since $m([0,1])=\infty$, $ \mu \times \nu ( D) \geq \infty$ because if it were less than $\infty$ we have some countable collection of rectangles $\{A_i \times B_i\}$ such that $D \subseteq \bigcup_1^\infty (A_i \times B_i)$ but $\cup B_i$ countable which contradicts $[0,1]$ uncountable.

This doesn't contradict the Fubini-Tonelli theorem because the counting measure is not $\sigma$-finite.
\end{pf}

\item Let $X=Y=\N$, $\M=\sN=\sP(\N)$ and $\mu=\nu=$ counting measure. Define \[ f(m,n)= \left\{
\begin{array}{ll}
1 & \text{ if } m=n\\
-1 & \text{ if } m=n+1 \\
0 & \text{ otherwise }
\end{array}
 \right. \]
 \[
\text{Prove that } \int |f|d(\mu \times \nu) = \infty \text{ and that } \int \int f d \mu d \nu, \int \int f d \nu d \mu \text{ exist and are unequal. }
\]
\begin{pf}
Let $D_1= \{(m,n) \in \N^2 : m = n\}$, $D_2= \{ (m,n) \in \N^2 : m=n+1 \}$ and $D = D_1 \cup D_2$. \begin{eqnarray*}
	\text{ Then, } f=\chi_{D_1}-\chi_{D_2}, |f| = \chi_{D}. \text{ Notice }
 \int |f|d(\mu \times \nu) & = & \int \chi_{D} d(\mu \times \nu) \\
	& \geq & \int \chi_{D_1}d(\mu \times \nu)  \\
	& = & (\mu \times \nu)(D_1)  \\
	& = & (\mu \times \nu) \bigcup\{(m,n) \in \N^2 : m=n \}   \\
	& = & \sum (\mu \times \nu) \{(m,n) \in \N^2 : m=n \}   \\
	& = & \sum \mu(m) \nu(n)    \\
	& = & \sum 1  = \infty. 
\end{eqnarray*}
\begin{eqnarray*}
	\int f d\mu d \nu & = & \int \chi_{D_1}-\chi_{D_2} d\mu d \nu \\
	& = & \int \mu (\chi_{D_1}-\chi_{D_2})^n \chi_\N d\nu  \\
		& = & \int  \chi_{\N}(\mu({D_1}^n)-\mu({D_2}^n) ) d\nu  \\
	& = & \int  \chi_{\N}(\mu(\{n\})-\mu(\{n\}) ) d\nu     \\
	& = & \int  \chi_{\N}(1-1 ) d\nu =0  
\end{eqnarray*}
To calculate $\int f d\nu d \mu$, first note $D_{1_m}= \{ m\}$ if $m=n$ and $\O$ otherwise whereas $D_{2_m}=\{m\}$ if $m=n+1$ and $\O$ otherwise. Hence, if $(m,n) \in D_2$, $m \neq 0$. Thus, 
	\begin{eqnarray*}
	\int f d\nu d \mu & = & \int \chi_{D_1}-\chi_{D_2} d\nu d \mu \\
	& = & \int \mu (\chi_{D_1}-\chi_{D_2})^n \chi_\N d\nu  \\
		& = & \int  \chi_{\N}\nu({D_1}_m)-\chi_{\N \slash \{0\}}\nu({D_2}_m)  d\mu  \\
	& = & \int  \chi_{\N}\nu(\{m\})-\chi_{\N \slash \{0\}}\nu(\{m\})  d\mu  \\    
	& = & \int  \chi_{\N}- \chi_{\N \slash \{0\}}d\nu \\
	& = & \int  \chi_{\N\slash (\N \slash \{0\})}d\nu \\
	& = & \int  \chi_{\{0\}}d\nu = \nu(\{0\}=1
\end{eqnarray*}
\end{pf}

\end{enumerate}