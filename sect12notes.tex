
\begin{dfn}[algebra] An algebra $A$ is a nonempty collection of subsets of a nonempty set $X$ which is closed under finite unions and complements.
\end{dfn}

\begin{example} If $X = \{ 1, 2, 3 \}$ then the following are algebras:
\begin{itemize}
\item $ \{ \emptyset, X \} $,
\item $ \{ \emptyset, \{ 1 \}, \{ 2, 3 \}, X \}$,
\item $ \{ \emptyset, \{ 2 \}, \{ 1, 3 \}, X \} $,
\item $ \{ \emptyset, \{ 3 \}, \{ 1, 2 \}, X \}$, and 
\item $ P(X)$.
\end{itemize}
Any other subset of the power set is not an algebra.
\end{example}

\begin{dfn}[$\sa$] An algebra $A$ is a $\sigma$-algebra if it is closed under countable unions. \end{dfn}

\begin{example}
Any algebra over a finite set is a $\sigma$-algebra, as is the power set of the set $X$. More examples are found in exercises.
\end{example}
\begin{dfn}[co-countable sets] a cocountable subset of a set X is a subset Y whose complement in X is a countable set. In other words, Y contains all but countably many elements of X. While the rational numbers are a countable subset of the reals, for example, the irrational numbers are a cocountable subset of the reals.
\end{dfn}

\begin{dfn}[$\sa$ of co-countable or countable sets] The $\sigma$-algebra of co-countable or countable sets is the collection of subsets such that either the subset or its complement is countable.
\end{dfn}

\begin{prop} Given any nonempty collection $ \mathcal{E}$ in the power set of $X$, there is a smallest $\sigma$-algebra that contains $\mathcal{E}$. 
\begin{proof} The idea of the proof is to show that an arbitrary intersection of $\sigma$-algebras is a $\sigma$-algebra.  One then considers the collection of $\sigma$-algebras that contain $ \mathcal{E}$, which is nonempty since the power set is such a set, and then take the intersection of all such sets.
\end{proof}
\end{prop}
\begin{lem} If a collection of sets $\mathcal{E}$ is contained in a $\sigma$-algebra $\mathcal{B}$ then the $\sigma$-algebra generated by $\mathcal{E}$ is contained in $ \mathcal{B}$. \end{lem}

\begin{dfn}[$\sigma$-algebra generated by $\mathcal{E}$] If $\mathcal{E}$ is a subset of $P(X)$ then the $\sigma$-algebra generated by $\mathcal{E}$ is the smallest $\sigma$-algebra that contains $\mathcal{E}$.
\end{dfn}

\begin{dfn}[Borel $\sigma$-algebra] The Borel $\sigma$-algebra for a metric space $X$ (the $\sigma$-algebra of Borel sets) is the $\sigma$-algebra generated by open sets.
\end{dfn}

\begin{dfn}[$G_{\delta}, F_\sigma$] A set in a metric space is a $G_{\delta}$ set if it is the countable intersection of open sets, it is an $F_{\sigma}$ set if it the countable union of a closed set.
\end{dfn}

\begin{example} $[0,1)$ is an $F_{\sigma}$ set since $ [0,1) = \cup [0, 1- \frac{1}{n}]$.  Any singleton is a $G_{\delta}$ set, it is also an $F_{\sigma}$ set.
\end{example}

\begin{prop} Each of the following sets generates the the Borel sets in $ \mathbb{R}$:
\begin{itemize}
\item open intervals,
\item closed intervals,
\item half open intervals,
\item open rays,
\item closed rays.
\end{itemize} 
\end{prop}

\begin{dfn}[product $\sa$] If $M_{\alpha}$ is a indexed collection of $\sigma$-algebras on an indexed collections of sets $X_{\alpha}$, then the product $\sigma$-algebra on the product space $X = \prod X_{\alpha}$ is the $\sigma$ algebra generated by the inverse image of elements of $M_{\alpha}$ under the coordinate maps $\pi_{\alpha}$.\\
The $\sa$ of subsets of $X \times Y$ generated by the semi-algebra $\sR$ is called the product $\sa$ and is denoted by $\A \otimes \B$.
\end{dfn}

\begin{example}
If $X_1, X_2$ are the indexed set with associated $\sigma$-algebras $M_1, M_2$ then the inverse image of the coordinate maps are sets of the form $ E_1\times X_2$ and $X_1 \times F_2$, where $E_1 \in M_1$ and $F_2 \in M_2$.  Notice that using intersections you get all sets of the form $E_1 \times F_2$.
\end{example}

\begin{prop} The product $\sigma$-algebra (in the case that the index set is countable) is generated by all products of things in the $M_i$'s
\end{prop}

\begin{prop} If each $M_{\alpha}$ is generated by a set $E_{\alpha}$ then the product sigma algebra is generated by elements of the $E_{\alpha}$, rather than arbitrary elements of $M_{\alpha}$. \end{prop}

\begin{thm} If $X_1, X_2, \cdots, X_n$ are metric spaces then the Borel sets on the product space is the product of the Borel sets on the individual spaces.
\end{thm}

\begin{proof}
The proof uses the preceding proposition.
\end{proof}

\begin{dfn}[elementary family] An elementary family is a collection of subsets that contains the empty set, is closed under finite intersections, and the complement of any set in the collection is a finite disjoint union of elements of the collection. \end{dfn}

\begin{example} Any $ \sigma$-algebra is an elementary family. 
In general the point of elementary families is starting with minimal information how does one build a $\sigma$-algebra, vs. knowing that there is a minimal $\sigma$-algebra.\end{example}

\begin{prop} The finite disjoint union of members of an elementary family is an algebra. \end{prop}
