\subsection{September 2 Group Assignment}

For the following problems, let $X$ be a nonempty set and let $E\subseteq \Ps (X)$. Note: I used the following resource on these problems: \url{http://math.stackexchange.com/questions/612266/on-sigma-algebra-generated-by-mathcale}. 

\begin{enumerate}
\item Let $\F$ denote the collection of countable subsets of $E$. If $F \in \F$ prove that the $\sa$ generated by $F$ is contained in the $\sa$ generated by $E$.

\begin{pf}
Let $F \in \F$, then $F \subset E$. Thus, by lemma 1.1,  $\M(F) \subset \M(E)$.   
\end{pf}

\item We will denote by $M_F$ as the $\sa$ generated by $F$. Prove that $N:= \bigcup_{F\in\F}M_F$ is a $\sa$.
\begin{pf}
$F \in N$ so $N \neq \O$. Suppose $Y \in N$. Then, $Y \in \M_F$. Since $\M_F$ is a $\sa$, $Y^c \in \M_F \subset N$. Thus, $Y^c \in N$. Consider $\{ {Y_i} \}_{i=1}^\infty \subset N$. Then, $Y_i \in M_{F_i}$ for some $i$. Consider $K = \bigcup_{i=1}^\infty F_i$. Notice $K$ is a countable union of countable subsets of $E$, so $K$ is a countable subset of $E$ and so $K \subset \F$ and $K \in M_\F \subset N$.  Then, for all $F_i$, $F_i \subset K$ so by lemma 1.1 in Rolland, $\M_{F_i} \subset \M_K \subset N$. Recall for all $Y_i \in \{ Y_i \}_{i=1}^\infty$, there exist $F_i$ such that $Y_i \in M_{F_i}$. Thus, for all $i$, $Y_i \in N$, so $\bigcup_{i=1}^\infty Y_i \subset N$. Hence, $N$ is closed under countable unions.  \end{pf}

\item Prove that $E$ is contained in $N$ and hence the $\sa$ generated by $E$ is contained in $N$.
\begin{pf}
 Let $x \in E$. Then, consider a countable sub collection of E that contains $x$, namely, $S_x$, such that $S_x = \bigcup_{x\in S \subset E}x$. Then, since $S_x$ is a countable subset of $E$, $S_x \in \F$, so $S_x \subset N:= \bigcup_{F\in\F}M_F$.   Notice $\bigcup_{x \in E} S_x=E$. From part (2), we know $N$ is a $\sa$, so if  $S_x \subset N$, $\bigcup_{x \in E} S_x \subset N$ and so $E \subset N$. Thus, $\M(E)\subset N$.
\end{pf}

\item Conclude that $N$ is the $\sa$ generated by $E$.
\begin{pf}
 From (3), we know $\M(E)\subset N$. So, we will show $N \subset \M(E)$. From part (1), we know for any $F \in \F$, $\M(F) \subset \M(E)$. Thus, $\bigcup_{F\in\F}M_F \subset \M(E)$. So, $N \subset \M(E)$. Hence, $N = \M(E)$, so $N$ is the $\sa$ generated by $E$. 
\end{pf}

\end{enumerate}




 