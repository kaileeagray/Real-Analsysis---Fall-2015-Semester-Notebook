%\newtheorem{thm}{Theorem}
%\newtheorem{lem}{Lemma}
%\newtheorem{prop}{Proposition}
%\newtheorem{cor}{Corollary}
%
%\theoremstyle{definition}
%\newtheorem{dfn}{Definition}
%\newtheorem*{construction}{Construction}
%\newtheorem*{example}{Example}

%\newtheorem*{conjecture}{Conjecture}
%\newtheorem*{acknowledgement}{Acknowledgements}
%\newtheorem{rmk}{Remark}

\begin{dfn}[measure]
Let $X$ be a set equipped with a $\sigma$-algebra $\M$. A measure on $\M$	or on $(X, \M)$ is a function $\mu: \M \rightarrow [0, \infty]$ such that \\
i. $\mu(\O)=0$,\\
ii. \textbf{(countable additivity)} if $\{E_j\}_1^\infty$ is a sequence of disjoint sets in $\M$ then \[
\mu\left( \bigcup_1^\infty E_j \right)=\sum_1^\infty \mu(E_j)
\]
\end{dfn}

\begin{dfn}[measurable space]
If $X$ is a set and $\M \subset \mathcal{P}(X)$ is a $\sa$, $(X, \M)$ is called a measurable space and the sets in $\M$ are called measurable sets. If $\mu$ is a measure on $(X, \M)$, then $(X, \M, \mu)$ is called a measure space.
\end{dfn}



\begin{dfn}[finite measure] Let $(X, \M, \mu)$ be a measure space. $\mu$ is called finite when $\mu(X)< \infty$. Note for all $E\in \M$, $\mu(X)=\mu(E)+ \mu(E^c)$, so for all $E \in \M$, $\mu(E)< \infty$. \end{dfn}

\begin{dfn}($\sigma-$finite measure) Let $(X, \M, \mu)$ be a measure space. $\mu$ is $\sigma$-finite if
	\[ X = \bigcup_{1}^\infty E_j \text{ where } \mu(E_j)<\infty \text{ for all } i, j.
	\]
\end{dfn}
\begin{dfn}($\sigma-$finite set) Let $(X, \M, \mu)$ be a measure space. $E$ is $\sigma$-finite for $\mu$ if
	\[ E = \bigcup_{1}^\infty E_j \text{ where } E_j \in \M, \ \mu(E_j)<\infty \text{ for all } i, j.
	\]
\end{dfn}

\begin{dfn}(semi-finite measure) Let $(X, \M, \mu)$ be a measure space. $\mu$ is semi-finite if for each $E \in \M$ with $\mu(E)=\infty$ there exists $F \in \M$ with $F \subset E$ and $0< \mu(F)< \infty$.
\end{dfn}


\begin{example}[measure] Let $X\neq \O$, $\M=\sP(X)$ and $f: X \rightarrow [0, \infty]$. $f$ determines a measure $\mu$ on $\M$ by the formula $\mu(E)=\sum_{x \in E}f(x)$.
	\begin{enumerate}
\item $\mu$ is semifinite iff $f(x)<\infty\ \forall x \in X$	
\item $\mu$ is $\sigma$-finite iff $\mu$ is semifinite and  $\{ x: f(x)>0\}$ is countable 
\item \textbf{(counting measure)} $f(x)=1$ for all $x$, then $\mu$ is counting measure
\item \textbf{(point mass, Dirac measure)} $f$ defined by $f(x_0)=1$ and $f(x)=0$ for all $x\neq x_0$, then $\mu$ is the point mass or Dirac measure
\end{enumerate}
\end{example}

\begin{example}[measure]
Let $X$ be an uncountable set and let $\M$ be the $\sa$ of countable or co-countable sets. The function $\mu$ on $\M$ defined by $\mu(E)=0$ if $E$ is countable and $\mu(E)=1$ if $E$ is co-countable is a measure.
\end{example}
\begin{example}[finitely additive measure, not a measure]
Let $X$ be an infinite set and $\M = \sP(X)$. Define $\mu(E)=0$ if $E$ is finite and $\mu(E)=\infty$ if $E$ is infinite. $\mu$ is a finitely additive measure but not a measure.	
\end{example}
\begin{thm}
Let $(X, \M, \mu)$ be a measure space.
\begin{enumerate}
\item \textbf{(monotonicity)}	If $E, F \in \M$ and $E \subset F$, then $\mu(E) \leq \mu(F)$.
\item \textbf{(subadditivity)}	If $\{E_j\}_1^\infty \subset \M$, then $\mu(\cup_1^\infty E_j) \leq \sum_1^\infty \mu(E_j)$. 
\item \textbf{(continuity from below)}	If $\{E_j\}_1^\infty \subset \M$ and $E_1 \subset E_2 \subset \dots, $ then $\mu(\cup_1^\infty E_j) = \lim_{j \rightarrow \infty} \mu(E_j)$. 
\item \textbf{(continuity from above)}	If $\{E_j\}_1^\infty \subset \M$ and $E_1 \supset E_2 \supset \dots, $ then $\mu(\cap_1^\infty E_j) = \lim_{j \rightarrow \infty} \mu(E_j)$.
\end{enumerate}
\end{thm}
\begin{dfn}[null set] If $(X, \M, \mu)$ is a measure space, $E \in \M$ is called null if $\mu(E)=0$.
\end{dfn}

\begin{rmk}
	Any countable union of null sets is a null set.
\end{rmk}

\begin{dfn}[almost-everywhere (a.e.)] If a statement is true about points $x \in X$ except for $x$ in some null set, then it is true almost everywhere, a.e.
\end{dfn}

\begin{rmk}
	If $\mu(E)=0$ and $F \subset E$, then $\mu(F)=0$ by monotonicity provided that $F \in \M$. If $F \not\in \M$ this may not be true.
\end{rmk}
\begin{dfn}(complete measure)
	A measure whose domain includes all subsets of null sets. This can always be achieved by enlarging the domain of $\mu$ as we can see in the theorem below.
\end{dfn}
\begin{thm}
	Suppose that $(X, \M, \mu)$ is a measure space. Let $\sN=\{N \in \M : \mu(N)=0\}$ and $\overline{\M}=\{E \cup F : E \in \M $ and $ F \subset N$ for some $N \in \sN\}$. Then, $\overline{\M}$ is a $\sa$ and there is a unique extension $\overline{\mu}$ of $\mu$ to a complete measure on $\overline{\M}$.
\end{thm}



