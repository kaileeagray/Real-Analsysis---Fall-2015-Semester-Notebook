This is exercise 23 on page 32 of Rolland.
\begin{enumerate}
\item Let $A$ be the collection of finite unions of sets of the form $(a, b] \cap \Q$ where $-\infty \leq a < b \leq \infty$.
\subitem(a) Prove that $A$ is an algebra on $\Q$.
\begin{pf}
Without loss of generality, assume $\left\{\bigcup_{i=1}^n (a_{i_j}, b_{i_j}] \cap \Q \right\}_{j=1}^N$ is a collection of disjoint sets so that for all $i, j$ we have $a_1 < a_2 < a_3 < \cdots$, $b_1 < b_2 < b_3 < \cdots$, and $b_1< a_2$, $b_2<a_3, \cdots$. Then,
\[
\bigcup_{j=1}^N\left( \bigcup_{i=1}^n (a_{i_j}, b_{i_j}] \cap \Q  \right)=\bigcup_{j=1}^N \bigcup_{i=1}^n \left((a_{i_j}, b_{i_j}] \cap \Q  \right)
\]
This is a finite union of sets in $A$, so $A$ is closed under finite unions. Also,
\[
\left( \bigcup_{i=1}^n (a_{i_j}, b_{i_j}]  \right)^c \cap \Q=\bigcap_{i=1}^n (a_{i_j}, b_{i_j}]^c \cap \Q=\left(\bigcap_{i=1}^n (-\infty, a_{i_j}] \cup (b_{i_j}, \infty) \right) \cap \Q. 
\] 
\[
\text{Notice we can rewrite } \bigcap_{i=1}^n (-\infty, a_{i_j}] \cup (b_{i_j}, \infty) =(-\infty, a_1]\cup (b_1, a_2] \cup (b_2, a_3] \cup  \cdots \cup (b_{n-1}, a_n] \cup (b_n, \infty). 
\]
The sets $(-\infty, a_1], (b_1, a_2], (b_2, a_3],  \cdots, (b_{n-1}, a_n], (b_n, \infty)$ are in $A$. So by the previous rewrite and since $A$ is closed under finite unions, $A$ is closed under complements. 
\end{pf}
\subitem(b) Prove that the $\sa$ generated by $A$ is the power set of $\Q$.
\begin{pf}
	Consider any $q \in \Q$. Then, $q = \cap_{n=1}^\infty (q - \frac{1}{n}, q]$. Any power set of $\Q$ is a countable union of these sets. Thus, any power set of $\Q$ is in $A$. Any set in $A$ is a set of rational numbers so it must be contained in the power set of $\Q$. 
\end{pf}
 \subitem(c) Define $\mu_0$by $\mu_0(\o)=0$ and $\mu_0(E)= \infty$ if $E \neq \o$. Prove $\mu_0$ is a pre measure on $A$. 	
 \begin{pf}
 Since $\mu_0(\O)=0$ we need only show that $\mu_0$ is additive over finite, disjoint unions. Consider $\{E_i\}_{i=1}^n$ with $E_i \cap E_j = \O$ for all $i \neq j$. Then, by definition of $\mu_0$, $\mu_0\left( \bigcup_{i=1}^n E_i \right)$ is either $0$ or $\infty$. Suppose $\mu_0\left( \bigcup_{i=1}^n E_i \right)=0$. Then, 	$\bigcup_{i=1}^n E_i= \O$ and so $E_i=\O$ for all $i$. Thus, $\mu_0(E_i)= 0$ for all $i$ and so $\sum_{i=1}^n \mu_0 (E_i ) = 0$. Hence, $\sum_{i=1}^n \mu_0( E_i)= \mu_0\left( \bigcup_{i=1}^n E_i \right)$.\\
 Suppose $\mu_0\left( \bigcup_{i=1}^n E_i \right)=\infty$. Then, 	$\bigcup_{i=1}^n E_i\neq \O$ and so there must exist at least one $E_i \neq \O$. Thus, $\mu_0(E_i)= \infty$ and so regardless of the other $E_i$'s $\sum_{i=1}^n \mu_0 (E_i ) = \infty$. Hence, $\sum_{i=1}^n \mu_0( E_i)= \mu_0\left( \bigcup_{i=1}^n E_i \right)$. Thus, $\mu_0$ is a pre-measure. 
 \end{pf}

 \subitem(d) Prove that there is more than one measure on $P(\Q)$ whose restriction to $A$ is $\mu_0$.
Take $\mu_1$ to be the measure induced by $\mu_0$. Then $\mu_1(A)=\infty$ unless $A=\emptyset$. Let $\mu_2$ be the counting measure. Now $\mu_1(\{1\})=\infty >1=\mu_2(\{1\})$. However, ${\A}_0$ consists of $\emptyset$ and infinite sets. Therefore $\mu_1|_{{\A}_0}=\mu_2|_{{\A}_0}$. 
 \subitem(e) Why does this not contradict Theorem 1.14?\\
Because $\mu_0$ is not $\sigma$-finite. 
\end{enumerate}
