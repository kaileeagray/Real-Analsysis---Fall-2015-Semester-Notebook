\newcounter{itemcounter3}
\begin{list}
{\text{(\arabic{itemcounter3})}}
{\usecounter{itemcounter3}\leftmargin=1.4em}
\item Let $(X, \M, \mu)$ be a measure space and $f: X \rightarrow \R$ be a measurable function, finite at every $x \in X$. Let $G_f= \{ (x, t) \in X \times \R : t = f(x) \}$ be the graph of $f$. If $\mu$ is $\sigma$-finite, prove that $G_f$ has measure zero with the product measure $\mu \times m$. Hint: use Fubini.
\begin{pf}
	We will show for $G_f= \{ (x, t) \in X \times \R : t = f(x) \}$, $(\mu \times m)(G_f)=0$. Since $\mu$ is $\sigma$-finite we can use Fubini to rewrite $(\mu \times m)(G_f)$ as 
	\[
	(\mu \times m)(G_f) = \int_{X \times \R} \chi_{G_f} d (\mu \times m) = \int_X \int_\R \chi_{G_f}\ dm \ d\mu. \text{ Also, note }
	\]
	\[
	(\chi_{G_f})_x  =  \left\{
	\begin{array}{ll}	
	1 & \text{ if } f(x) = y \\
	0 & \text{ if } y \neq f(x)
	\end{array} \right.\text{(eq. 1), and }
	\]
	\[
	({G_f})_x  =  \{y\} \text{ where } y = f(x), 
	\text{ so } m(({G_f})_x)  =  m(\{y\})=0. \ \text{(eq. 2)}
	\]
	Thus, by eq. 1 and 2, we have
	\begin{eqnarray*}
(\mu \times m)(G_f)& =&	\int_X \int_\R \chi_{G_f}\ dm \ d\mu\\
 & = & \int_X  \int_\R (\chi_{G_f})_x  m(y)   \mu(x)	\\
	& = & \int_X m({G_f}_x) \ d \mu(x) \\
	& = & \int_X 0 \ d \mu(x) \\
	& = & 0.
	\end{eqnarray*}
\end{pf}

\item Let $X=[0,1]$, $\M=B_{[0,1]}$, $m$ the Lebesgue measure on $\M$ and $\mu$ the counting measure on $\M$. Show that $m \ll \mu$, but that there is no function $f$ such that $dm=fd\mu$. Does this contradict the Radon-Nikodym theorem?
\begin{pf}
	For $E \in \M$ suppose $\mu(E)=0$. Then, since $\mu$ is the counting measure, $E$ must not contain any elements and so $E = \O$. Hence, $m(E)=0$ and $m \ll \mu$. \\Next, suppose there is an $f$ such that $dm = f d \mu$. Let $a \in [0, 1]$ and note $m(\{ a \})=0$ and 
	\[
	m(\{a\}) = \int_{\{a\}} f d \mu = f(a) \mu (\{ a \}) = f(a). \qquad \text{ Hence, } f(a) = 0 \quad \forall \ a \in [0,1].
	\]
	Also, notice $m([0,1])=1$ but 
	\[
	m([0,1]) = \int_{[0,1]}f d \mu = \int 0 d \mu = 0 \mu([0,1]) = 0.
	\]
	Thus, there is no $f$ such that $dm = f d \mu$.\\
	This does not contradict the Radon-Nikodym theorem because the counting measure is not $\sigma$-finite.
\end{pf}

\item (a) Let $f_n: [1, \infty) \rightarrow \R$ be a function defined by $f_n(x)=\frac{1}{x}\chi_{[n, \infty)}(x)$. Show that the sequence $\{f_n\}$ converges to zero uniformly on $[1, \infty)$. 
\begin{pf} Let $f_n: [1, \infty) \rightarrow \R$ be a function defined by $f_n(x)=\frac{1}{x}\chi_{[n, \infty)}(x)$. Note \[
f_n(x) = \left\{
\begin{array}{ll}
\frac{1}{x} & \text{ if } x \geq n \\
0 & \text{ if } x < n
\end{array} \right. .
\]
Thus, when $f_n(x) = \frac{1}{x}$, $f_n(x) \leq \frac{1}{n}$. So, for any $\varepsilon>0$ pick $N$ such that $\varepsilon > \frac{1}{N}$. Then, for all $n \geq N$ and for all $x$, $|f_n (x)|\leq \frac{1}{n} \leq \frac{1}{N}< \varepsilon$. Hence, $\{f_n\}$ converges to zero uniformly on $[1, \infty)$.
\end{pf}

(b) State Fatou's lemma. \\
If $\{f_n\}$ is any sequence in $L^+$, then 
	\[
	\int(\liminf f_n) \leq \liminf \int f_n.
	\]
\\
(c) Apply Fatou's lemma to the sequence from part (a).\\
By corollary to Fatou's Lemma, since $f_n \rightarrow 0$
\[
\int 0 =0 \leq \liminf \int \frac{1}{x}\chi_{[n, \infty)}(x) \ dm = \int_{[n, \infty)} \frac{1}{x} \ dm.\]
\item Let $f_n: X \rightarrow \R$ be measurable, bounded functions such that for every $n \in \R$, $x \in X$, $f_n(x) \geq f_{n+1}(x)$ and there is a measurable function $f: X \rightarrow \R$ such that $\lim f_n(x)=f(x)$ pointwise. If $\int f_k \ d \mu < \infty$ for some $k \in \N$, prove that 
\[
\lim_{n \rightarrow \infty} \int f_n d \mu = \int f d\mu.
\]
\item Prove the following:
(a) If $f$ is monotonic, then $f$ is Lebesgue measurable. 
\begin{pf}
	Let $f: \R \rightarrow \R$ be monotone.  By Proposition
2.3 in Page 44, it suffices to show that for any $a \in \R$, we have $f^{−1}((a, \infty))$ is Borel measurable. WLOG, assume $f$ is increasing. Let $x^{'}=\inf \{x : f(x) >a  \}$\\
	\textbf{Case 1:	Suppose $f(x^{'}) \leq a$.}
 We will show $f^{-1}((a, \infty))=(x^{'}, \infty)$. First, show $f^{-1}((a, \infty))\subseteq (x^{'}, \infty)$. Let $x \in f^{-1}((a, \infty))$. Then, $f(x)>a$. Since $x^{'}=\inf \{x : f(x) >a  \}$, $x^{'}<x$. Thus, $x \in (x^{'}, \infty)$. \\
	Next, show $f^{-1}((a, \infty))\supseteq (x^{'}, \infty)$. Let $x \in (x^{'}, \infty)$. Then, $x>x^{'}$.  Since $x^{'}=\inf \{x : f(x) >a  \}$ and $x>x^{'}$, there exists some $x_0 \in \R$ such that $x>x_0>x^{'}$ and $f(x_0)>a$. $f$ is monotone, so $f(x)>f(x_0)$. Thus, $f(x)>a$ which implies $x \in f^{-1}((a, \infty))$.  \\
\textbf{Case 2: Suppose $f(x^{'}) > a$}. We will show $f^{-1}((a, \infty))=(x^{'}, \infty)$. First, show $f^{-1}((a, \infty))\subseteq (x^{'}, \infty)$. Let $x \in f^{-1}((a, \infty))$. Then, $f(x)>a$. Since $x^{'}=\inf \{x : f(x) >a  \}$, $x^{'}<x$. Thus, $x \in (x^{'}, \infty)$. \\
	Next, show $f^{-1}((a, \infty))\supseteq (x^{'}, \infty)$. Let $x \in (x^{'}, \infty)$. Then, $x>x^{'}$.  Since $f$ is monotone, so $f(x)>f(x^{'})> a$. Thus, $f(x)>a$ which implies $x \in f^{-1}((a, \infty))$.  \\
	\textbf{Case 3: Suppose $f(x^{'}) =\infty$}. We will show $f^{-1}((a, \infty))=\O$. If $f(x^{'}) =\infty$, $f(\inf\{x : f(x)>a\}) =\infty$ which implies $\inf\{x : f(x)>a\} =\infty$ so $\{x : f(x)>a\}=\O$. Thus, $f^{-1}((a, \infty))=\O$.\\
	\textbf{Case 4: Suppose $f(x^{'})=-\infty$}. We will show $f^{-1}((a, \infty))=\R$. If $f(x^{'}) =-\infty$, $f(\inf\{x : f(x)>a\}) =-\infty$ which implies $\inf\{x : f(x)>a\} =-\infty$ so $\{x : f(x)>a\}=\R$. Thus, $f^{-1}((a, \infty))=\R$.\\
	Hence, for any $a \in \R$, we have $f^{−1}((a, \infty))$ is Borel measurable, so $f$ is measurable. 
\end{pf}

(b) If $f$ is continuous and $g$ is Lebesgue measurable, then $f \circ g$ is Lebesgue measurable. 
\begin{pf}
Assume $f$ is continuous and $g$ is Lebesgue measurable. Note $(f \circ g)^{-1}=g^{-1} \circ f^{-1}$ and
$
(g^{-1} \circ f^{-1})((a,\infty )) = g^{-1}(f^{-1}(a,\infty )).
$
Since $f$ is continuous, $f^{-1}((a,\infty))$ must be an interval, so $f^{-1}((a,\infty )) \in \B_\R$. Then, since $g$ is Lebesgue measurable,  $g^{-1}(f^{-1}(a, \infty)) \in \sL $. Hence, $f \circ g$ is Lebesgue measurable. 
\end{pf}

(c) If $f$ is continuous and $g$ is Lebesgue measurable, is $g \circ f$ Lebesgue measurable? \\
No. Consider the following counterexample: \\


\item Let $f(x) = \int_0^\infty e^{-xt}(t^{-3}\sin^3(t))dt$. Show that
(a) $f(x)$ is well-defined for each $x \in [0, \infty)$. 
\begin{pf}
Consider any $x \in [0, \infty)$. Then, \[
|f(x)| \leq \int_0^\infty |e^{-xt}(t^{-3})\sin^3t| dt \leq \int_0^\infty t^{-3} < \infty. \text{ So, } f(x) \text{ exists for any } x \in [0, \infty).
\]	
Next, suppose $x_1, x_2 \in [0, \infty)$ such that $x_1 = x_2$. Then, 
\begin{eqnarray*}
f(x_1) - f(x_2) & = & \int_0^\infty e^{-x_1t}(t^{-3})\sin^3t \ dt - \int_0^\infty e^{-x_2t}(t^{-3})\sin^3t \ dt \\
& = & \int_0^\infty (e^{-x_1t}(t^{-3})\sin^3t - e^{-x_2t}(t^{-3})\sin^3t) \ dt \\
& = & \int_0^\infty (e^{-x_1t}- e^{-x_2t})(t^{-3})\sin^3t \ dt \\
& =& 0.
\end{eqnarray*}
Hence, $f$ is well-defined.
\end{pf}
(b) $f(x)$ is continuous on $[0, \infty)$.
\begin{pf}
Let $x_n \rightarrow x$. Then, from part a, we know $|e^{-x_nt}(t^{-3})| \leq t^{-3}$, so by the dominated convergence theorem, 
\begin{eqnarray*}
\lim_{n \rightarrow \infty}f(x_n) & = & \lim_{n \rightarrow \infty}	 \int_0^\infty e^{-x_nt}(t^{-3})\sin^3t dt \\
& = &  \int_0^\infty \lim_{n \rightarrow \infty}	 e^{-x_nt}(t^{-3})\sin^3t dt \\
& = &  \int_0^\infty  e^{-xt}(t^{-3})\sin^3t dt \\
& = & f(x).
\end{eqnarray*}
	Thus, $f(x)$ is continuous on $[0, \infty)$. 
\end{pf}

\item Suppose that $f$ is a continuous real-valued function of bounded variation on $[0,1]$ and that for each $\epsilon \in (0,1)$, $f$ is absolutely continuous on $[\epsilon, 1]$. Must $f$ necessarily be absolutely continuous on $[0,1]$?
\item Suppose $f \in L^1[0,1]$ satisfies 
\[
\int_E |f| \leq (m(E))^2 \text{ for every measurable set } E \subseteq [0,1]. 
 \]
\text{ Show that $f$ is a.e. equal to zero.} Use Lebesgue differentiation theorem.
\item Prove or give a counterexample: every dense open subset of $(0,1)$ has Lebesgue measure 1.
\item Let $f \in L^1(\R)$ such that $f(x)=0$ for $|x|\geq 1$. Prove that $f_n$ defined by $f_n(x)=f\left(x+ \frac{1}{n} \right)$ converges to $f$ in $L^1(\R)$. Is the condition that $|x|\geq 1$ necessary?
\end{list}