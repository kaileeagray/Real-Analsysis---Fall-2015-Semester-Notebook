\newcounter{itemcounter2}
\begin{list}
{\text{(\arabic{itemcounter2})}}
{\usecounter{itemcounter2}\leftmargin=1.4em}
\item Let $f$ be real-valued.\\
(a) Give the definition of a measurable function.\\
(p. 43, Folland) If $(X, \M)$ and $(Y, \sN)$ are measurable spaces, a mapping $F: X \rightarrow Y$ is called $(\M, \sN)$-measurable if $f^{-1}(E) \in \M$ for all $E \in \sN$.
(b) If $f$ is measurable, is $f^2$ measurable? Justify.\\
Yes. From group work, we know $fg$ is measurable. So, $ff$ is measurable. 

(c) If $f^2$ is measurable, is $f$ measurable. Justify.\\
use counterexample of $f=\chi_{N_r}-\chi_{N_r^c}$. Note $f$ is not measurable but $f^2 = 1$ since $f=\pm 1$

\item Let $X$ be a countable set and $\mu$ a measure on $X$. Assume that for any $F \subseteq X$ there is $G \subseteq F$, $G \neq \O$, with $\mu(G)< \infty$. Prove the following:\\
 (a) For any $x \in X$, $\mu(\{x \})< \infty$.\\
 (b) $\mu$ is $\sigma$-finite.\\
 \begin{pf}
 Note $X = \cup_{x \in X}\{x\}$. By part a $\mu(\{x\})< \infty$. Thus, $X$ can be written as a countable union sets where each set has finite measure. Therefore, $\mu$ is $\sigma$-finite.
 \end{pf}
 \item Let $(X, \M, \mu)$ be a measure space. Assume $\{f_n\}$ is a sequence in $L^1(X, \mu)$ so that $f_n \rightarrow f$ pointwise, and there exists $M>0$ so that for all $n$ and all $x \in X$ the inequality $|f_n(x)|\leq M$ holds.\\
 (a) If $\mu(X)< \infty$, show that $f \in L^1(X, \mu)$ and $\int f_n \rightarrow \int f$.\\ 
 $M$ dominated convergence theorem - \\
 (b) Show by example that the above conclusion may fail if $\mu(X)=\infty$.
 \item Let $\mu^*$ be an outer measure on a set $X$. If $E \subseteq X$ satisfies $\mu^*(E)=0$ prove that $E$ is $\mu^*-$measurable. 
\begin{pf}
	
\end{pf} 
 \item Let $f:[0,1]\rightarrow \R$ be a continuous function of bounded variation such that $f(0)=0$. \\
 (a) Give an example of such a function for with the identity 
 \[
 f(x)=\int_0^x f^{'}(x)dx \text{ fails to hold for a.e. } x \in [0,1].
 \]
 (b) For what type of functions $f$ does the identity in (a) hold almost everywhere?
 \item (a) State the Monotone Convergence Theorem.\\
 If $\{f_n\}$ is a sequence in $L^+$ such that $f_j \leq f_{j+1}$ for all $j$, and $f = \lim_{j \rightarrow \infty}f_n$, then $\int f = \lim_{n \rightarrow \infty} \int f_n$.\\
 (b) State Fatou's Lemma.\\
 If $\{f_n\}$ is any sequence in $L^+$, then 
	\[
	\int(\liminf f_n) \leq \liminf \int f_n.
	\]
 (c) Prove Fatou's Lemma using the Monotone Convergence Theorem.
 \begin{pf} (Proof from Folland, p. 52) \\
 For each $k \geq 1$ we have $\inf_{n \geq k} f_n \leq f_j$ for $j \geq k$. Hence $\int \inf_{n \geq k} f_n \leq \int f_j$ for $j \geq k$. Therefore, $\int \inf_{n \geq k}f_n \leq \inf_{j \geq k} \int f_j$. Let $k \rightarrow \infty$ and apply the monotone convergence theorem to obtain: \[
 \int (\liminf f_n) = \lim_{k \rightarrow \infty}\int \inf_{n \geq k} f_n \leq \liminf \int f_n.
 \]	
 \end{pf}

  \item Define $F: \R \rightarrow \R$ by 
 \[ F(x) = \left\{
 \begin{array}{ll}
	0 & \text{ if } x< \frac{1}{4},\\
	x & \text{ if } \frac{1}{4}\leq  x< 1,\\
	x^2+1 & \text{ if }  x\geq 1,\\
\end{array} \right.
 \]
 Let $\mu_F$ be the Borel measure associated to $F$.\\
 (a) Calculate $\mu_F((\frac{1}{4},1])$ and $\mu_F([\frac{1}{4},1])$.\\
 By exercises from October 9, we know $\mu_F([a,b))=F(b-)-F(a-)=\mu_F((a, b])=F(b) - F(a)$. Thus,
 \[
 \mu_F\left(\left(\frac{1}{4},1\right]\right) = F(1) - F\left( \frac{1}{4} \right) = 1^2 + 1 - \frac{1}{4} = \frac{7}{4}. 
 \]
 \[
  \mu_F\left(\left[\frac{1}{4},1\right)\right) = F(1-) - F\left( \frac{1}{4} -\right) = 1-0 = 1. 
 \]
 (b) Calculate the Lebesgue derivative of $\mu_F$. \\
 To calculate, recall the Lebesgue-Radon-Nikodym representation given in Theorem 3.22, $d \mu_F = d \lambda + f \ dm$ where $f$ is the Lebesgue derivative and since we are working in $\R$ we can write
 \[
 f(x) = \lim_{r \rightarrow 0} \frac{\mu_F(B(r,x))}{m(B(r,x))} = \lim_{r \rightarrow 0}\frac{\mu_F((x-r,x+r))}{m((x-r,x+r))} = \lim_{r \rightarrow 0} \frac{F((x+r)-)-F(x-r)}{2r}.
 \]
 Suppose $x < \frac{1}{4}$, then\[
 \lim_{r \rightarrow 0} \frac{F((x+r)-)-F(x-r)}{2r} = \lim_{r \rightarrow 0} \frac{0-0}{2r} =0.
 \]
  Suppose $ \frac{1}{4}< x < 1$, then\[
 \lim_{r \rightarrow 0} \frac{F((x+r)-)-F(x-r)}{2r} = \lim_{r \rightarrow 0} \frac{x+r - (x-r)}{2r} =1.
 \]
   Suppose $ x>1$, then\[
 \lim_{r \rightarrow 0} \frac{F((x+r)-)-F(x-r)}{2r} = \lim_{r \rightarrow 0} \frac{(x+r)^2 + 1 - ((x-r)^2 +1) }{2r} =\lim_{r \rightarrow 0} \frac{4rx}{2r} = 2x.
 \]
 Suppose $x = \frac{1}{4}$, then\[
 \lim_{r \rightarrow 0} \frac{F\left(\left(\frac{1}{4}+r\right)-\right)-F\left(\frac{1}{4}-r\right)}{2r} = \lim_{r \rightarrow 0} \frac{\frac{1}{4}+r}{2r} = \infty \quad (r>0)
 \]
  Suppose $x = 1$, then\[
 \lim_{r \rightarrow 0} \frac{F\left(\left(1+r\right)-\right)-F\left(1-r\right)}{2r} = \lim_{r \rightarrow 0} \frac{(1+r)^2 + 1 - (1-r)}{2r} = \lim_{r \rightarrow 0} \frac{1+3r+r^2}{2r} = \infty \quad (r>0)
 \]
 (c) Give the Lebesgue-Radon-Nikodym decomposition of $\mu_F$ with respect to $m$. 
 \item Let $(X, \M, \mu)$ and $(Y, \sN, \nu)$ be measures spaces and assume that $E \in \M \times \sN$. Show that if $\mu \times \nu (E)=0$ then $\nu(E_x)=\mu(E^x)$ for a.e. $x$ and $y$. 
 \begin{pf}
 	
 \end{pf}

\end{list}