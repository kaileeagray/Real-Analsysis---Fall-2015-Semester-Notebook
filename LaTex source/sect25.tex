%\newtheorem{thm}{Theorem}
%\newtheorem{lem}{Lemma}
%\newtheorem{prop}{Proposition}
%\newtheorem{cor}{Corollary}
%
%\theoremstyle{definition}
%\newtheorem{dfn}{Definition}
%\newtheorem*{construction}{Construction}
%\newtheorem*{example}{Example}

%\newtheorem*{conjecture}{Conjecture}
%\newtheorem*{acknowledgement}{Acknowledgements}
%\newtheorem{rmk}{Remark}
%get notes from lect. 24: http://textofvideo.nptel.iitm.ac.in/111101005/lec25.pdf
Let $(X, \A)$ and $(Y, \B)$ be measurable spaces. 
\begin{dfn}[measurable rectangles] A subset $E \subseteq X \times Y$ is called a measurable rectangle if $E=A \times B$ for some $A \in \A$ and $B \in \B$. \\
Let $\sR$ denote the class of all measurable rectangles. $\sR$ is not, in general, a $\sa$. It is a semi-algebra of subsets of $X \times Y$. 
\end{dfn}
\begin{rmk}
\[
(A \times B) \cap (E \times F) = (A \cap E) \times (B \cap F), \qquad (A \times B)^c = (X \times B^c) \cup (A^c \times B)
\]	
Thus, the collection $\A$ of finite disjoint unions of rectangles is an algebra. 
\end{rmk}

\begin{dfn}
$\M \otimes \N$ is the smallest $\sa$ generated by rectangles, $A \times B, A \in \M, B \in \sN$. 	
\end{dfn}
\begin{dfn}
$\mu \times \nu: \M \otimes \sN \rightarrow [0, \infty]$. $(\mu\times \nu)(A\times B)=\mu(A)\nu(B)$.	
\end{dfn}
\begin{rmk}
If $\mu$ and $\nu$ are $\sigma$-finite, say 	
\end{rmk}

\begin{dfn}
$E \in \M \otimes \sN$. Fix $x$, $E^x= \{ y: (x,y)\in E \}\subseteq Y$. Fix $y$, $E_y=\{x: (x,y) \in E\} \subseteq X$.	
\end{dfn}
\begin{dfn}
$f: X \times Y \rightarrow \R$. $f^x: Y \rightarrow \R$, $f^x(y)=f(x,y)$. $f_y: X \rightarrow \R$, $f_y(x) = f(x,y)$	
\end{dfn}
\begin{thm}
If $f$ is $\M \otimes \sN$ measurable, then $f^x$ is $\sN$ measurable and $f_y$ is $\M$ measurable.	
\end{thm}

\begin{rmk}
$f \in L^+ $ implies $f$ measurable and non-negative	
\end{rmk}
\begin{thm}[Tonelli (positive functions)]
Let $(X,\M, \mu)$ and $(Y, \sN, \nu)$ be $\sigma$-finite measure spaces. 
If $f \in L^+$, then $g(x)=\int f_x d \nu \in L^+(\nu)$ and $h(y)=\int f^y d \mu \in L^+(\mu)$. Key point: 
\[
\int f d (\mu \times \nu)=\int \left( \int f(x,y) d\mu\right)d \nu = \int \left( \int f(x,y) d\nu\right)d \mu. \text{ So, } \int h(y)d\nu=\int g(x) d\mu
\]
\end{thm}
\begin{thm}[Fubini (integrable functions)]
Let $(X,\M, \mu)$ and $(Y, \sN, \nu)$ be $\sigma$-finite measure spaces. 
If $f\in L^{1}(\mu \times \nu$, then $f_x \in L^1(\nu)$ a.e. $x$ and $f^y \in L^1(\mu)$ a.e. $y$, then
\begin{eqnarray*}
	g(x) = \int f_x d \nu \in L^1 (\mu) \\
	h(y) = \int f^y d \mu \in L^1(\nu) \text{ and} \\
	\int f d(\mu \times \nu)=\int \left[ \int f(x,y)d\mu \right] d \nu \\
	=\int \left[ \int f(x,y) d \nu \right] d \mu 
\end{eqnarray*}
\end{thm}

