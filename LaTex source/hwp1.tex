\newcounter{itemcounter}
\begin{list}
{\text{(\arabic{itemcounter})}}
{\usecounter{itemcounter}\leftmargin=1.4em}

\item Let $\A$ be an algebra of sets that is closed under countable increasing unions. Show that $\A$ is a $\sigma$-algebra.
\begin{pf}
Let $\A$ be an algebra on some set $X$ that is closed under countable increasing unions. Then, consider a countable collection of sets in $\A$, namely $\{E_i\}_{i=1}^\infty$. To show $\A$ is a $\sa$ we must show $\bigcup _{i=1}^\infty E_i \in \A$. Define $F_n = \bigcup _{j=1}^n E_j$. Since $\A$ is an algebra, $\A$ is closed under finite unions, so $F_n \in \A$ for all $n$. Further, $\{F_j \}_{j=1}^\infty \subset \A$. Additionally, notice $F_1=E_1$, $F_2 = E_1 \cup E_2$, $F_3 = E_1 \cup E_2 \cup E_3, \cdots$ so $F_1 \subseteq F_2 \subseteq F_3 \subseteq \cdots$. Thus, $\{F_j \}_{j=1}^\infty$ is an increasing collection of sets. Because $\A$ is closed under countable increasing unions, $\bigcup_{j=1}^\infty F_j \in \A$. But $\bigcup_{j=1}^\infty F_j = \bigcup _{i=1}^\infty E_i $. Thus, $\bigcup _{i=1}^\infty E_i \in \A$ so we have shown $\A$ is a $\sa$. 	
\end{pf}

\item Let $A\subset E \subset B$, where $A, B$ are Lebesgue measurable sets of finite measure. Prove that if $m(A)=m(B)$, then $E$ is measurable.
\begin{pf}
	Let $A\subset E \subset B$, where $A, B$ are Lebesgue measurable sets of finite measure. Assume $m(A)=m(B)$. Notice if $A \subset E \subset B$ set subtraction implies $\O \subset E \backslash A \subset B \backslash A$. Since $m(A)=m(B)$ and $m(A), m(B)$ are finite we can write $m(B)-m(A)=0$ which implies $m(B \backslash A) = 0$. Thus, $B \backslash A$ is a set of measure zero. Since Lebesgue measure is complete, if $E \backslash A$ is a subset of a set of measure zero, $E \backslash A$ must be measurable. $E = E\backslash A \cup A$, so $E \backslash A$ and $A$ measurable imply $E$ measurable since the collection of measurable sets is a  $\sa$. 
\end{pf}


\item If $\{f_n \}_{n\in \N}$ is a sequence of Lebesgue measurable real-valued functions, prove that \\ $f=\liminf f_n$ is Lebesgue measurable. 
\begin{pf}
	
\end{pf}

\item Let $f \colon \R \rightarrow [0, \infty)$ be Lebesgue measurable. 
\begin{enumerate}[(a)] % (a), (b), (c), ...
\item Let $E_m=\{ x \in \R \colon f(x)> 1/m \}$. Use the monotone convergence theorem to show
\begin{equation*}
 \lim_{m \rightarrow \infty} \int_{E_m} f dm = \int_{\R} f dm.
\end{equation*}
\begin{pf}
	
\end{pf}
\item Prove that if $\int_\R fdm<\infty$, then for all $\varepsilon < \infty$, there exists $A \in \B_\R$ with $m(A)<\infty$ so that  
\begin{equation*}
  \int_{\R} f dm < \int_{A} f dm + \varepsilon.
\end{equation*}
\begin{pf}
	Assume $\int_\R fdm<\infty$. Then, $f \in L^+$ and \[
	\int_\R f \ d m = \sup \left\{ \int_\R  \phi\ d m : 0 \leq \phi \leq f, \ \phi \text{ simple} \right\}.
	\] 
	Thus, for any $\epsilon>0$, we can find a simple function $\phi$ such that \[
	 \int_\R f \ dm < \int_\R \phi \ dm + \epsilon 
	\]
	 Let \[
	\phi = \sum_1^n a_j \chi_{E_j}. \text{ Then, } \int \phi \ dm = \sum_1^n a_j m(E_j).
	\]
	Since $\int f \ dm $ is the supremum of all such integrals and $\int f \ dm < \infty$, $\int \phi \ dm < \infty$. Thus, if there exists some $E_j$ with $m(E_j) = \infty$, the corresponding $a_j = 0$. So, consider only $E_j$ with $m(E_j) \neq \infty$. Thus, by exercise 4 from September 18, for every $E_j$ with $m(E_j)< \infty$ there exists some $G_{\delta}$ set, $A_j$, where $m(E_j)=m(A_j)$. Since $m(E_j)< \infty$, $m(A_j) < \infty$. Next, let
	\[
	A= \bigcup_1^k A_j \quad E = \bigcup_1^k E_j
	\]
	so $A_j$ $G_\delta$ implies $A \in \B_R$. Also, $m(A_j) < \infty$ for all $j$ implies $m(\cup_1^k A_j) = m(A) < \infty$.  Additionally, since $a_j = 0$ whenever $m(E_j) = \infty$, we can write
	\[
	 \int_\R \phi \ dm = \sum_1^n a_j m(E_j) = \sum_1^k a_j m(E_j) = \sum_1^k a_j m(A_j)= \int_A \phi \ dm.
	\]
	Then, since $\int_A f \ dm = \sup \{ \int_A \phi \ dm : 0 \leq \phi \leq f, \phi $ simple $ \}$, we have $\int_A \phi \ dm \leq \int_A f \ dm$. Hence, 
	\[
	 \int_\R f \ dm < \int_\R \phi \ dm + \epsilon = \int_A \phi \ dm + \epsilon \leq \int_A f \ dm + \epsilon. 
	\]

\end{pf}
\end{enumerate} 


\item Let $\{f_n \}$ be a sequence of Lebesgue integrable functions that converge to $f$ in $L^1$. 

\begin{enumerate}[(a)] % (a), (b), (c), ...
\item Prove that $\{ f_n \}$ converges to $f$ in measure.
\begin{pf}
Assume $\{f_n \}$ is a sequence of Lebesgue integrable functions that converge to $f$ in $L^1$. Then, $\int |f_n - f| \rightarrow 0$. Let $E_{n, \epsilon}=\{x: |f_n(x)-f(x)|\geq \varepsilon\}$
**see page 61, prop 2.29**
\end{pf}
\item Give an example of a sequence $\{f_n\}$ and a function $f$ such that $\{f_n\}$ converges to $f$ in measure, but $\{f_n\}$ does not converge to $f$ in $L^1$. 
\\
$f_n= n\chi_{[0, \frac{1}{n}]}$\\
\[
\{x: n\chi_{[0, \frac{1}{n}]}\geq \varepsilon \} \subset \{ x: n\chi_{[0, \frac{1}{n}]}>0 \}
\]
$\mu(\{ x: n\chi_{[0, \frac{1}{n}]}>0 \})=\frac{1}{n}$\\
\[
\int n \chi_{[0, \frac{1}{n}]} = n\mu\left(\left[0, \frac{1}{n}\right]\right)=1
\]
\end{enumerate} 


\item Let $f, g$ be Lebesgue integrable functions on $\R$. Prove that the function $F(x,y)=f(y)g(x-y)$ is Lebesgue integrable in $\R^2$.
\begin{pf}
	Since $f,g$ are Lebesgue integrable, $fg$ is measurable. So, $\int|f| < \infty$ and $\int|g| < \infty$. Also, by Tonelli, $f(y)g(x-y) \in L^+$, and\[
	\int |f(y)g(x-y)| d(m \times m)(x,y) = \int \int |f(y) g(x-y)| dm(x) dm(y).
	\]
Then, since $f(y)$ does not depend on $x$ we can write
\[
\int \int |f(y) g(x-y)| dm(x) dm(y)= \int |f(y)| \int |g(x-y)| dm(x) dm(y).
	\]
	Also, Lebesgue integration is translation invariant so $\int g(x-y)$ is the same as $\int g(x)$. Thus,
	
	\[
	\int |f(y)| \int |g(x-y)| dm(x) dm(y) = \int |f(y)| dm(y) \int |g(x)| dm(x) .
	\]

\end{pf}

\item Let $\mu_F$ be the Borel measure on $\R$ with distribution function
\[
F(x) = 
\begin{cases}
  \arctan (x) & \text{if}\ x<0, \\
  x^2 +1          & \text{if}\ \geq 0
\end{cases}
\]
\begin{enumerate}[(a)] % (a), (b), (c), ...
\item Calculate $\mu_F([0,3))$ and $\mu_F((0,3))$.\\
By exercises from October 9, we have $\mu_F([0, 3)) = F(3-) - F(0-) = 3^2 + 1 - \arctan(0) = 10$. Also, $\mu_F((0, 3)) = F(3-) - F(0) =3^2 +1 - (0^2 + 1) = 9$.
\item 
If $\mu(E)=0$ for every $E \in \R$ with $m(E)=0$, then we write $\mu \ll m$ and say $\mu$ is absolutely continuous with respect to Lebesgue measure.

\item Prove that $\mu_F$ is not absolutely continuous with respect to Lebesgue measure.
\begin{pf}
	Notice 
	\[
	\mu_F(\{0\}) = \mu_F([0,3) \slash (0,3)) = \mu_F([0,3)) - \mu_F ((0,3)) = 10-9 =1
	\]
	but $m(\{0\})=0$. This, $\mu_F \not \ll m$. 
\end{pf}

\end{enumerate}

\item Construct a family of Lebesgue measurable functions $\chi_t \colon \R \rightarrow \R$, $t \in \R$, with the property that $\chi = \sup _{t\in \R} \chi_t $ is not a Lebesgue measurable function. (You may assume without the proof that non-measurable sets exist. 
\\
(exercise 6, $\S 2.1$): Show by example that there is an uncountable set $A$ and for each $a \in A$ a measurable function $f_a$, but $\sup \{f_\alpha : \alpha \in A\}$ is not measurable.  \\	
Consider the set $N_r$ constructed in section 1.1.  Then $N_r$ is an uncountable set and therefore not measurable. However, for every $r \in \Q\cap[0,1)$ and $x \in N$ (where $N$ was defined as the subset of $[0,1)$ containing exactly one member of the equivalence classes defined by $x \sim y$ iff $x-y \in \Q$. Singletons are measurable, so, from page 46 of Folland, the indicator functions $\chi_{\{x+r\}}$ and $\chi_{\{x_r-1\}}$ are measurable for all $r \in \Q\cap [0,1)$ and $x \in N\cap [0,1-r)$ or $x \in N\cap [1-r,1)$. Notice $\sup\{ \chi_{\{x+r\}}, \chi_{\{x+r-1\}}: r \in \Q\cap [0,1)$ and $x \in N\cap [0,1-r)$ or $x \in N\cap [1-r,1) \}=\chi_{N_r}$ But, $ \chi_{N_r}$ is not measurable because $N_r$ is not measurable. 
\item Show by way of an example that an open, dense set in $\R$ need not have infinite measure.


\item Let $f$ be a continuous function of bounded variation. Prove that $f=f_1-f_2$ where both $f_1, f_2$ are monotonic and continuous. 

\end{list}
