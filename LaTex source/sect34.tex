\begin{rmk}
From now on, we will focus on $(X, \M, \mu)= (\R^n, \B_{\R^n}, m)$. 	
\end{rmk}
\begin{rmk}
$B(r, x) = \{ y : ||x - y||<r \}$, $m(B(r,x))= k(n) \ \pi r^n$.	
\end{rmk}
\textbf{Our goal: }We want $\nu \ll m$ so the RN theorem gives us $f d m = d \nu$ and \[
f = \frac{d \nu}{d m} \text{ we will make this precise: } \lim \frac{\nu(...)}{m(...)}
\]
\begin{lem}[Vitali's Covering Lemma]
	$\C$ is a collection of balls in $\R^n$ and $\mathcal U = \cup_{B \in \C} B$. If $c< m(\mathcal U)$ there are $B_1, B_2, \dots, B_k \in \C$ disjoint such that 
	\[
	\sum_{i=1}^k m(B_i) > \frac{c}{3^n}.
	\]
\end{lem}
How to think about Vitali's theorem: there is some $\mathcal U$ and take a disjoint subset (maximized). However, you will miss stuff. The theorem says if you take all these subsets and blow them up by a factor of 3, then you are guaranteed to cover $\mathcal U$. \\
We need $c$ because there may be unbounded stuff.
\begin{rmk}
In other words, $\C = \{ B(r,x) \}_{\{r,x\} \subset r,x}$	
\end{rmk}

\begin{pf}[Vitali's lemma proof]
Step 1: Find a compact set $K \subseteq \U$ with 	$m(K) > c$. \\
Step 2: Because $K$ is compact there is $A_1, A_2, \dots, A_m \in \C$ with $K \subseteq \cup_{j=1}^m A_j$. In other words, there exists a finite subcovering. Definition of compact: every cover has a finite subcover. \\
Step 3: Now we have finite but not disjoint. So start with $\{A_1, A_2, \dots, A_m \}$ and choose the largest. Call it $B_1$. With what's left, chose the largest that doesn't hit $B_1$ (is disjoint from $B_1$) call $B_2$. If there is a tie, just pick one. With what's left... call it $B_3$...\\
We will end with $B_1, \dots, B_n$ disjoint. \\
Step 4: Now we need to show 
\[
	\sum_{i=1}^k m(B_i) > \frac{c}{3^n}.
	\]
	Let $B_i = B(r_i, x_i)$. $A_j \subseteq B(3r_i, x_i)$ for some $i$. Assume $A_j \neq B_i$ for any $i$. Then $A_j \cap B_k \neq \O$ for some $k$. Choose the smallest $k$ for which $A_j \cap B_k \neq \O$. 
	See figure. \\
	So $A_j \cap B_1 = \O, A_j \cap B_2 = \O, \dots, A_j \cap B_{k_0 -1}= \O$ radius $A_j \leq $ radius for $B_{k_0}$.
	$x \in A_j \cap B_{k_0}$, $d(x_{k_0}, y)< r_{k_0}$\\
	$d(y, x')< r_{k_0}$, and $d(x', z) < r_{k_0}$ for any $z \in A_j$. So $A_j \subseteq B(3r_i, x_i)$ for some $i$. \\
	$c < m(K) \leq m(\cup_{k=1}^n(B(x_k, 3r_k)))$
	\[
	K \subseteq \bigcup_{j=1}^m A_j \subseteq \bigcup_{k=1}^n B(x_k, 3r_k)
	\]
	So, $m(\cup_{k=1}^n(B(x_k, 3r_k))) \leq \sum_{k=1}^n m (B(x_k, 3r_k))=3^n \sum_{k=1}^n B(x_k, r_k)$. 
\end{pf}

\begin{dfn}
$f$ is locally integrable if 
\[
\int_K |f|d m < \infty \text{ for any compact set } K \subseteq \R^n. 
\]	
Let $L_{loc}^{1} (m)$. Let $f(x) \in L^{1}_{loc}$.
\end{dfn}
Example, average value function of $f$ on $B(r,x)$: $$ (A_r f )(x) = \frac{1}{m(B(r,x))} \int_{B(r,x)} f(y) dm(y)$$
Example, Hardy-Littlewood Maximal Function: Pick a function $f$ and fix $x$
$$
(Hf)(x) = \sup_{r>0}(A_r |f|)(x) = \sup_{r>0} \frac{1}{m(B(r,x))} \int_{B(r,x)} |f(y)| d m (y)
$$

Facts: (use dominated convergence theorem to prove)
\[
1. \ \lim_{r \rightarrow r_0} (A_r f)(x) = (A_{r_0}f)(x); \qquad \lim_{x \rightarrow x_0}(A_r f)(x) = (A_r f)(x_0)
\]
$$\text{ Notice  }\int_{B(r,x)}f(y)dm(y) = \int_{\R^n} \chi_{B(r,x)}f(y) dm(y)$$
2. $Hf(x)$ is measurable \\
3. There is $c> 0$ such that for all $f \in L^1$ and $\alpha > 0$
\[
m(\{ x: Hf(x)> \alpha \}) \leq \frac{c}{\alpha} \int |f| d m
\]
\begin{pf}[proof of 3]
	Let $E_\alpha = \{ x: Hf(x) > \alpha \}$. Then, for all $x$, there is an $r_x>0$ such that $(A_{r_x}|f|(x)) > \alpha$. \\
	Consider $\cup_{x \in E_{\alpha}} \supseteq E_{\alpha}$. Then, for any $c < m(E_{\alpha})$, 
	\[
	c < 3^n \sum_{j=1}^k m(B_j) \text{ for some finite } B_1, \dots, B_j. \text{ Then, }
	\] 
	\begin{eqnarray*}
	c & < & 	3^n \sum_{j=1}^k m(B_j) \\
	& \leq & \frac{3^n}{\alpha} \sum{j=1}^k \int_{B_j} |f| d m \\
	& \leq &  \frac{3^n}{\alpha} \int_{\R^n} |f| d m
	\end{eqnarray*}
	Let $c \rightarrow m(E_{\alpha})$. 
	Read proof of differentiation theorem. He will hand out problem for next time. 
 \end{pf}
 
 \begin{thm}
$$
f \in L^1_{loc} \Rightarrow \lim_{r \rightarrow 0} A_r f(x) = f(x) \ \ \text{ a. e. }
$$
\end{thm}
\begin{pf}
1. True if the function is cts.	
\[
\frac{1}{m(B(r,x))} \int_{B(r,x)}f d m
\]
Let $\varepsilon>0$ there exists $\delta$ s.t. $y \in B( \delta, x)$ implies $f(x) - \varepsilon < f(y) < f(x) + \varepsilon$\\
2. Approximate $f$ by cts function $g$. \\
3. (key step)
\[
\limsup_{r \rightarrow 0} |A_r f(x) - f(x)|
\]
If this is 0, then this is a limit, get rid of $||$. \\
Look at the set were $\limsup \neq 0$ and show this set has measure zero.
\begin{eqnarray*}
	\limsup_{r \rightarrow 0} |A_r f(x) - f(x)| & = & \limsup_{r \rightarrow 0} |A_r (f - g)(x) + A_r g(x) - g(x) + g(x) - f(x)| \\
	& \leq & |H(f-g)(x)| + |0| + |(g-f)(x)|
\end{eqnarray*}
4. $E_{\alpha} = \{ x: \limsup |A_r f(x) - f(x)|> \alpha \}$. \\
$F_{\alpha}=\{ x : |f- g|> \alpha \}$. \\
$E_{\alpha} \subseteq F_{\alpha \slash 2} \cup \{ x: H(f-g) > \frac{\alpha}{2} \}$
$$m(E_{\alpha} \leq m(F_{\alpha \slash 2}) + m\left( \left\{ x: H(f-g) > \frac{\alpha}{2} \right\} \right) \leq \frac{\epsilon}{2} + \frac{c \epsilon}{\frac{\alpha}{2}} $$
Thus, $m(E_{\alpha}) = 0 \ \forall \ \alpha$.
\end{pf}

\begin{dfn}[Lebesgue set]
\[
L_f = \left\{  x: \lim_{r \rightarrow 0} \frac{1}{m(B(r,x))} \int_{B(r,x)} |f(y) - f(x)| dm(y) = 0  \right\}
\]	
\end{dfn}

\begin{cor}
\[
m(L_f^c) = 0
\]	
\end{cor}

\begin{rmk}[notation]
$\{E_r\}_{r>0}$ is a collection of sets. Then, $E_r$ shrinks nicely to $x$ if \\
1. $E_r \subseteq B(r,x)$ for all $r$\\
2. There is a fixed $\alpha$ such that \[
m(E_r)> \alpha m(B(r,x))
\]
\end{rmk}

\begin{thm}[Lebesgue Diffn Thm]
	If $f \in L^1_{loc}$ and $E_r$ shrinks nicely to $x \in L_f$, then
	\[
	\lim_{r \rightarrow 0} \frac{1}{m(E_r)} \int_{E_r}|f(y) - f(x)|dy = 0.
	\]
\end{thm}

\begin{dfn}
$\nu$ is a Borel measure on $\R^n$. $\nu$ is regular if \\
i. $\nu(K)< \infty$ for any compact $K$ \\
ii. [follows from i.] $\nu(E) = \inf \{ \nu(U) \ : \ U \text{ open } E \subseteq  U\}$ for any Borel set $E$. \\
If $\nu$ is signed it is regular if $|\nu|$ is regular.
\end{dfn}

\begin{thm}
If $\nu$ is a regular signed measure with LRN representation $d \nu = d \lambda + f dm$ and 
 \[
f(x) = \lim_{r \rightarrow 0} \frac{\nu(E_r)}{m(E_r)} \text{ if } \{ E_r \} \text{ shrinks nicely to } \{ x \}.
\]	
\end{thm}




