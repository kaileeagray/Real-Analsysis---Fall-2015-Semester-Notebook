\begin{enumerate}
\item Let $M,N$ be $\sa$s on $X,Y$ respectively. We define a rectangle to be a set $A \times B \subseteq X \times Y$ such that $A \in M$ and $B \in N$.	
\subitem(a) Prove that the $\sa$ generated by the set of rectangles is equal to $M \otimes N$.
\begin{pf}
	Let $R$ denote the set of rectangles. Note $M \otimes N = $ set of all sets in $M \times Y \cup X \times$ set of all sets in $N$. We will show the $\sa$ generated by $R$, $\sR=M \otimes N$. First, show $\sR \subset M \otimes N$.
	For any $A \times B \in R, A \times B \subseteq X \times Y$ with $A \in M$ and $B \in N$. Then, $A \times B = (A \times Y) \cap (X \times B) \in M \otimes N$. 
	Next, show $\sR \supset M \otimes N$. Note $M \otimes N = $ set of all sets in $M \times Y \cup X \times$ set of all sets in $N$. So for any $M_i \times N_i \in M \otimes N$, $M_i \times N_i = (A \times Y) \cap (X \times B)$ for some $A \in M, B \in N$. Then, $M_i \times N_i = A \times B$ for some $A \in M$ and $B \in N$, so $M_i \times N_i \in \sR$. \end{pf}

\item Let $M,N$ be $\sa$s on $X,Y$ respectively. Given a set $E\subseteq X \otimes Y$ and a fixed $x\in X$, we define the $x$ cross-section
$
E_x = \{y:(x,y) \in E\}	
$
and for a fixed $y \in Y$ we define the $y$ cross-section $
E^y = \{x:(x,y) \in E\}$	
\subitem(a) Prove that if $E=A \times B$ is a rectangle, then 
 \[  E_x =  \left\{
\begin{array}{ll}
      B & x\in A \\
      \emptyset & x \notin A 
\end{array} \text{ and }
\right. \]
 \[  E^y =  \left\{
\begin{array}{ll}
      A & y\in B \\
      \emptyset & y \notin B 
\end{array} 
\right. \]
\begin{pf}
	Let $M,N$ be $\sa$s on $X,Y$ respectively. Given a set $E\subseteq X \otimes Y$, with $E = A \times B$ and a fixed $x\in X$, $E_x=\{y: (x,y)\in E\}$. For all $(x,y) \in E$, $y \in B$. So, $E_x=B$ when $x \in A$. If $x \not \in A$, then $(x,y) \not \in E$ so $E_x = \O$. Thus, 
	\[  E_x =  \left\{
\begin{array}{ll}
      B & x\in A \\
      \emptyset & x \notin A 
\end{array} 
\right. \]
Next, for a fixed $y\in Y$, $E^y=\{x: (x,y)\in E\}$. For all $(x,y) \in E$, $x \in A$. So, $E^y=A$ when $y \in B$ and $(x,y) \in E$. If $y \not \in B$, then, $(x,y) \not\in E$ so $E^y = \O$. Thus, 
\[  E^y =  \left\{
\begin{array}{ll}
      A & y\in B \\
      \emptyset & y \notin B 
\end{array} 
\right. \]
\end{pf}

\subsubitem(b) Consider $M=N=B_\R$ and let $E=\{(x,y) : x <y \}$. Determine 
\begin{equation*}
E^{\frac{1}{3}}, E_{\frac{1}{3}}	, E^0, E_1, \text{ and } E^{\frac{1}{2}}.
\end{equation*}
\begin{equation*}
\begin{multlined}
E^{\frac{1}{3}}=\left\lbrace x: \left(x, \frac{1}{3} \right) \in E \right\rbrace. \text{ This is the line } y=\frac{1}{3}. \\
E_{\frac{1}{3} } =\left\lbrace y: \left(\frac{1}{3}, y \right) \in E \right\rbrace. \text{ This is the line } x=\frac{1}{3}\\
E^0 = \left\lbrace x: (x, 0) \in E \right\rbrace. \text{ This is the line } y=0 \\
E_1= \{y : (1,y) \in E \}. \text{ This is the line}x=1 \\
E^{\frac{1}{2}}=\left\lbrace x: \left(x, \frac{1}{2} \right) \in E \right\rbrace. \text{ This is the line } y=\frac{1}{2}.
\end{multlined}
\end{equation*}

\subitem(c) Prove that the set $\sR$ consisting of $E \subseteq X \times Y$ such that $E_x\in N$ for all $x \in X$ and $E^y \in M$ for all $y \in Y$ contains all rectangles. 
\begin{pf}
	 Consider $\sR = \{ E \subset X \times Y: E_x \in N $ for all $x \in X$ and $E^y \in M $ for all $y \in Y\}$. $\sR$ contains all rectangles since all $x$ from $M \times N$ are either from $N$ or $\O$ and all $y$ are either from $M$ or $\O$. 
\end{pf}

\subitem(d) Prove that $\sR$ is a $\sa$, and hence it contains all of $M \otimes N$.
\begin{pf}
	Consider some $\{E_i\}_{i=1}^\infty$ with $E_i \subseteq X \times Y$ for all $i$. Then, $\bigcup_{i=1}^\infty E_i = (\cup E_i)_x \times (\cup E_i)^y$. By part (d), $E \in M \otimes N \subseteq \R$. So for all $E \in R$, $E_x \in N$ and $E_y \in M$. Notice $(\cup E_i)_x=\cup(E_i)_x$ and $(E_i)_x \in N$ implies $\cup(E_i)_x \in N$. $N$ is an $\sa$, so $(\cup E_i)_x \in N$. Similarly, $(\cup E_i)^y=\cup(E_i)^y$ and $(E_i)^y \in M$ implies $\cup(E_i)^y \in M$. $M$ is an $\sa$, so $(\cup E_i)^y \in M$. Thus, $\sR$ is closed under countable unions. Also, for any $E \in \sR$, $E_x \in N$ and $E^y \in M$. Since $M, N$ are $\sa$ $(E_x)^c = (E^c)_x$ and $(E^y)^c = (E^c)^y$ imply $(E^y)^c, (E_x)^c \in M, N$ respectively. Thus, $\sR$ is closed under complements. So, $\sR$ is a $\sa$
\end{pf}
\end{enumerate}
