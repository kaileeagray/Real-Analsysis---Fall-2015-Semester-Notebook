\begin{dfn}[absolutely continuous]
	$\nu$ is a signed measure and $\mu$ is a positive measure on $(X, \M)$. $\nu$ is absolutely continuous with respect to $\mu$ and write \[
	\nu \ll \mu
	\]

\end{dfn}

\begin{thm}[Lebesgue-Radon-Nikodym Theorem]
	$\nu$ is $\sigma$-finite signed measure on $(X, \M)$\\
	$\mu$ is $\sigma$-finite measure on $(X, \M)$\\
	then there are unique $\sigma$-finite signed measures, $\lambda, \rho$ on $(X, \M)$ s.t. 
	\begin{enumerate}
	\item $\lambda \perp \rho$
	\item $\rho \ll \lambda$
	\item 	$\nu = \lambda + \rho$
	\end{enumerate}
further there is $f \in L^1( \mu)$ s.t. $d \rho = f d \mu$ and $f$ is unique $\mu$-a.e.
\end{thm}

\begin{pf}[sketch of LRN proof]
	Case 1: $\mu$ and $\nu$ are finite measures. \\
	Step 1: Let $\F = \{ f:X \rightarrow [0, \infty] : \int_E f d \mu \leq \nu(E) $ for all $ E \in \M \}$.
	\begin{enumerate}
	\item $0 \in \F$ implies $\F \neq \O$
	\item 	$f, g \in \F$ then $\max \{ f,g \} \in \F$
	\begin{pf}. $A = \{ x: f(x) > g(x) \}$. 	$\int_E \max \{ f,g \} d \mu= \int_{E \cap A} f d \mu + \int_{E \backslash A} g d \mu \leq \nu(E \cap A) + \nu(E \backslash A) = \nu(E)$	
	\end{pf}
	\end{enumerate}
Step 2: Let $a = \sup \{ \int_X f d \mu: f \in \F\}$. Then, \\
(1) $a \leq \nu(X) < \infty$\\
(2) There is $f \in \F$ such that $\int_X f d \mu = a$ 
\begin{pf}[proof of (2)] Choose $f_n \in \F$ such that $\int_{n \rightarrow \infty} \int_X f_n d \mu = a$. \\
If $g_n = \max\{ f_1, f_2, \dots, f_n \} \in \F$, then $f(x) = \lim_{n \rightarrow \infty} g_n(x)$.\\ $g_{n + 1} \geq g_n$ and $g_n$ are always bounded.
\[
\int_X g_n d \mu \geq \int_X f_n d \mu. \text{ By MCT } \forall n, \int_X f_n d \mu \leq \int_X f d \mu  = a
\]
\[
\text{ Since } g_n \in \F,  \nu(E) \geq \lim \int_E g_n d \mu = \int_E f d \mu . \text{ Thus } f \in \F.
\]
\[
\text{So, } \int_X f d \mu \leq a. \quad \int_X f_n d \mu \leq \int_X g_n d \mu \leq \int_X f d \mu \ \forall n. \text{ Thus, } \int_X f d \mu = a \text{ and } f \in L^1( \mu).
\]
\end{pf}
What we know: $\rho(E) = \int_E f \mu$. Any measure defined in this will satisfy $\rho \ll \mu$.\\
Now, prove parts (1) and (3). \\
Let $\lambda(E) = \int_E d \nu - \int f d \mu = \nu(E) - \rho(E)$. Thus, $\lambda$ is a signed measure. And since $\nu, \rho, \lambda$ are finite, we have $\nu = \lambda + \rho$. \\
Next, we need to check $\lambda \perp \mu$. Suppose $\mu(E)> 0$. Then, \[
\lambda(E) = \nu(E) - \rho(E) = \nu(E) - \int_E f d \mu
\]
Suppose you have an $E$ with $\lambda(E) > 0$. Then, $\nu(E) > \int_E f d \mu$. Define a new function that is a little bigger than $f$ on $E$ by lemma. But, that would contradict $a = \sup$. 

We would need to deal with case 2, 3...
\end{pf}
\begin{rmk}
For this course, you should at least remember the following theorems: dominated convergence, Lebesgue-Radon-Nikodym, Fubini-Tonelli.
\end{rmk}

\begin{rmk}[notation]
	If $\nu$ is $\sigma$-finite and $\mu$ is $\sigma$-finite. Then, 
	$\nu= \lambda + \rho$ with $\lambda \perp \mu$, $\rho \ll \mu$.
	This is the Lebesgue Radon Nikodym decomposition of $\nu$ wrt $\mu$.
\end{rmk}
\begin{dfn}[LRN derivative of $\nu$ wrt $\mu$]
\[
d \nu = f d \mu. \ f = \frac{d \nu}{d \mu}. \text{  Then, } \rho(E)= \int_E  \frac{d \nu}{d \mu } d \mu = \int_E f d \mu
\]
\end{dfn}

\begin{dfn}
$\nu_1, \nu_2, \mu$ are $\sigma$finite signed measures.
\[
f_{\nu_1+\nu_2}= \frac{d(\nu_1 + \nu_2)}{d \mu}= \frac{d\nu_1}{d \mu}+\frac{d\nu_2}{d \mu}= f_{\nu_1}+ f_{\nu_2}
\]
\end{dfn}

\begin{thm}
If $\nu \ll \mu$, $\mu \ll \sigma$, then $\nu \ll \sigma$. \\
\[
\frac{d \nu}{d \mu} \cdot \frac{d \mu}{d \sigma} = \frac{d \nu}{d \sigma} \text{ and }  \nu(E) = \int_E \frac{d \nu}{d \sigma} d \sigma = \int_E \frac{d \nu}{d \mu} \frac{d \mu}{d \sigma}  d \sigma
\]	
\[
\nu(E) = \int_E \frac{d \nu}{d \mu} d \mu, \qquad \mu(E) = \int_E \frac{d \mu}{d \sigma} d \sigma 
\]
\end{thm}

\begin{rmk}
If $\nu \ll \mu, \mu \ll \nu$, then
\[
\frac{d \mu}{d \nu} \cdot \frac{d \nu}{d \mu}=1 \text{ a.e. }  
\]
\end{rmk}

\begin{thm}
Let $\nu$ be a finite signed measure and $\mu$ a positive measure on $(X, \M)$. Then, $\nu \ll \mu$ iff for every $\varepsilon>0$ there exists $\delta>0$ such that $|\nu(E)|< \varepsilon$ whenever $\mu(E)< \delta$. 	
\end{thm}

\begin{cor}
If $f \in L^1(\mu)$, for every $\varepsilon>0$ there exists $\delta>0$ such that $|\int_E f d \mu| < \varepsilon$ whenever $\mu(E)< \delta$.	 Denote
\[
\nu(E) = \int_E f d \mu: \  \ \ d \nu = f d \mu.
\]
\end{cor}




