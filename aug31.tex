\subsection{August 31 Group Assignment}
\begin{enumerate}
\item Prove proposition 1.2 from Folland (p. 22): $\B_\R$ is generated by each of the following:
 the open intervals: $\E_1 =\{ (a, b): a < b \} $,
 the closed intervals: $\E_2 = \{ [a,b]: a < b \}$, the half-open intervals: $\E_3 = \{ (a,b] : a < b \}$ or $\E_4 = \{ [a,b) : a < b \}$,
the open rays: $\E_5 = \{ (a,\infty) : a \in \R\}$ or $\E_6 = \{ (-\infty, a) : a \in \R\}$,
the closed rays: $\E_7 = \{ [a,\infty) : a \in \R\}$ or $\E_8 = \{ (-\infty, a] : a \in \R\}$.

\begin{pf}
Let $\SO$ denote the collection of all open intervals in $\R$ so that $\M(\SO)=\B_\R$. For all $j, j\neq 3,4$, the elements of $\E_j$ are Borel sets so $\E_j \subset \B_\R$ implies $\M(\E_j) \subset \B_\R$. So we will prove $\M(\E_j) \supset \B_\R$ for all $j \neq 3,4$. It will suffice to show $(a,b) \in \M(\E_j)$ for any $a, b \in \R$ with $a < b$ since this implies $\M((a,b)) \subset \M(\E_j)$ and so $\B_\R \subset \M(\E_j)$ as desired.\\
	\begin{enumerate}
\item[$\E_1$] Every open set in $\R$ is at most a countable union of open intervals, so for \\ $\E_1 =\{ (a, b): a < b \} $, $\M(\E_1) \supset \B_\R$. 
\item[$\E_2$] Prove the closed intervals: $\E_2 = \{ [a,b]: a < b \}$, generate $\B_\R$. Notice 
\[ (a,b)=\bigcup_{n=1}^\infty \left[a + \frac{1}{n}, b-\frac{1}{n} \right]. \text{ Thus, } (a,b) \in \M(\E_2).
\]
\item[$\E_3$] Prove the $\E_3 = \{ (a,b] : a < b \}$ generate $\B_\R$. Notice 
\[ (a,b)=\bigcup_{n=1}^\infty \left(a, b-\frac{1}{n} \right]. \text{ Thus, } (a,b) \in \M(\E_3). \text{ So } \B_\R \subset \M(\E_3)
\]
Also, notice 
\[ (a,b]=\bigcap_{n=1}^\infty \left(a, b+\frac{1}{n} \right). \text{ Thus, } (a,b] \in \B_\R. \text{ So }  \M(\E_3) \subset \B_\R. \text{ Therefore, } \M(\E_3)=\B_\R.
\]
\item[$\E_4$] Prove $\E_4= \{ [a,b) : a < b \}$, Notice 
\[ (a,b)=\bigcup_{n=1}^\infty \left[a-\frac{1}{n}, b \right). \text{ Thus, } (a,b) \in \M(\E_4. \text{ So } \B_\R \subset \M(\E_4).
\]
Also, notice 
\[ [a,b)=\bigcap_{n=1}^\infty \left(a+\frac{1}{n}, b \right). \text{ Thus, } [a,b) \in \B_\R. \text{ So }  \M(\E_4) \subset \B_\R. \text{ Therefore, } \M(\E_4)=\B_\R.
\]

\item[$\E_5$] Prove the open rays: $\E_5 = \{ (a,\infty) : a \in \R\}$ generates $\B_\R$. Notice 
\[ (a,b)=\bigcup_{n=1}^\infty \left(a + \frac{1}{n}, b-\frac{1}{n} \right]= \bigcup_{n=1}^\infty \left\lbrace \left(a+\frac{1}{n}, \infty \right) \cap \left(b - \frac{1}{n}, \infty \right)^c \right\rbrace 
\]
implies $(a,b) \in \M(\E_5)$. Thus, $\SO \subset \M(\E_5)$ which implies $\B_\R \subset \M(\E_5)$. 
\item[$\E_6$] Prove $\E_6 = \{ (-\infty, a) : a \in \R\}$ generates $\B_\R$. Notice 
\[ (a,b)=\bigcup_{n=1}^\infty \left(a + \frac{1}{n}, b-\frac{1}{n} \right]= \bigcup_{n=1}^\infty \left\lbrace \left(a+\frac{1}{n}, \infty \right) \cap \left(b - \frac{1}{n}, \infty \right)^c \right\rbrace 
\]
implies $(a,b) \in \M(\E_6)$. Thus, $\SO \subset \M(\E_6)$ which implies $\B_\R \subset \M(\E_5)$. 
\item[$\E_7$] Prove the closed rays $\E_7 = \{ [a,\infty) : a \in \R\}$ generates $\B_\R$. Notice 
\[ (a,b)=\bigcup_{n=1}^\infty \left[a - \frac{1}{n}, b+\frac{1}{n} \right)= \bigcup_{n=1}^\infty \left\lbrace \left[a-\frac{1}{n}, \infty \right) \cap \left[b + \frac{1}{n}, \infty \right)^c \right\rbrace 
\]
implies $(a,b) \in \M(\E_7)$. Thus, $\SO \subset \M(\E_7)$ which implies $\B_\R \subset \M(\E_7)$. 
\item[$\E_8$] Prove $\E_8 = \{ (-\infty, a] : a \in \R\}$ generates $\B_\R$. Notice 
\[ (a,b)=\bigcup_{n=1}^\infty \left(a + \frac{1}{n}, b-\frac{1}{n} \right]= \bigcup_{n=1}^\infty \left\lbrace \left(-\infty , b-\frac{1}{n} \right] \cap \left(-\infty, a + \frac{1}{n}\right]^c \right\rbrace 
\]
implies $(a,b) \in \M(\E_8)$. Thus, $\SO \subset \M(\E_8)$ which implies $\B_\R \subset \M(\E_8)$. 
\end{enumerate}
\end{pf}
\end{enumerate}




