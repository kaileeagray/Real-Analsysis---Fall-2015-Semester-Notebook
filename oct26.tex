\begin{enumerate}
\item Suppose $\{ f_n\}_{n=1}^\infty$ is a countable sequence of functions in $L^1(\mu)$ and that $f_n \rightarrow f$ uniformly.
\subitem(a) If $\mu(X)< \infty$, prove that $f \in L^1(\mu)$ and $\int f_n \rightarrow \int f$. 
\begin{pf}
Let $\epsilon>0$ and $g= \max\{|f_1|, |f_2|, \dots, |f_N|\}+2\epsilon$. We will show $|f_n|\leq g$	for any $n$. If $n \leq N$, then $|f_n|\leq \max\{|f_1|, |f_2|, \dots, |f_N|\}\leq g + 2\epsilon$ so suppose $n > N$. Since $f_n \rightarrow f$ uniformly, for any $\epsilon>0$, there exists an $M \geq 0$ such that $|f_n(x)-f(x)|< \epsilon$ for all $n>M$ and $x \in X$. Then, $|f_n|< |f| + \epsilon$. So, for $n=N$, we can write $|f|< |f_N| + \epsilon$. Thus,
\[
|f_n|< |f| + \epsilon<|f_N| + \epsilon+ \epsilon \leq g.
\]
Since $\int |f_i| d\mu < \infty$ for any $i$ and $\mu(X)<\infty$, let $\int |f_i| d\mu =C$ and $\mu(X)=K$, then  
\[
\int g d\mu = \int |f_i| d\mu + \int 2\epsilon d\mu =C+2\epsilon\mu(X)=C+2\epsilon K < \infty. 
\]
Thus, $g \in L^1$. Hence, by the dominated convergence theorem, $f \in L^1(\mu)$ and $\int f_n \rightarrow \int f$. 
\end{pf}

\subitem(b) Find an example to prove that the previous can be false if $\mu(X) = \infty$. 
 \[
\text{Let } f_n =\frac{1}{2n} \mathlarger{\chi}_{\ _\mathlarger{[-n,n]}} .\quad \text{ Then, } \mu([-n,n])=\infty. \text{ Note, } \int f_n = \frac{1}{2n}(2n)=1. 
\]
Also, $f_n \rightarrow 0$ and $\int 0 = 0$. However $\int f_n = 1$ but $\int f = 0$ so 1(a) does not apply if $\mu(X) = \infty$. 
\item \[ \text{Let } f(x) = \left\{
\begin{array}{ll}
x^{-\frac{1}{2}} & \text{ if } 0 < x< 1\\
0 & \text{ otherwise} 	
\end{array} \right.\ .
\]
 
If $\{r_n\}$ is an enumeration of the rationals, set 
\[
g(x)=\sum_{n=1}^\infty 2^{-n}f(x-r_n). \text{ Prove that } g \in L^1(m) \text{ but } g^2 \text{ is not.}
\] 	
\begin{pf}
\[\begin{array}{lll}
\text{Notice } \mathlarger{\int}\limits_{(r_j, 1+ r_j)}(x-r_1)^{-\frac{1}{2}}dm=&\mathlarger{\int} (x-r_1)^{-\frac{1}{2}} \mathlarger{\chi}_{\ _\mathlarger{(r_j, 1+ r_j)}}dm& \\
& = \mathlarger{\int} (x-r_1)^{-\frac{1}{2}} \lim\limits_{n \rightarrow \infty}\mathlarger{\chi}_{\ _\mathlarger{(r_j+ \frac{1}{n}, 1+ r_j)}}dm & \\
& = \lim\limits_{n \rightarrow \infty}\mathlarger{\int} (x-r_1)^{-\frac{1}{2}} \mathlarger{\chi}_{\ _\mathlarger{(r_j+ \frac{1}{n}, 1+ r_j)}}dm & \\
\end{array}
\]
\[
\text{Note that } f_j = f(x-r_j) \mathlarger{\chi}_{\ _\mathlarger{(\frac{1}{n}+r_j, 1+r_j)}}=\left\{
\begin{array}{ll}
(x-r_j)^{-\frac{1}{2}}& \text{if } r_j + \frac{1}{n} < x < 1+r_j\\
0 & \text{ otherwise} 	
\end{array}
\right.. 
\]	
Thus, $f_j < \sqrt{n}$. So, by theorem 2.28, $f_j$ is Lebesgue measurable and the Riemamn integral is equal to the Lebesgue integral on an interval. Thus,  

\[\begin{array}{ll}
 \lim\limits_{n \rightarrow \infty}\mathlarger{\int} (x-r_1)^{-\frac{1}{2}} \mathlarger{\chi}_{\ _\mathlarger{(r_j+ \frac{1}{n}, 1+ r_j)}}dm =& \lim\limits_{n \rightarrow \infty} \mathlarger{\int\limits}_{(r_j+ \frac{1}{n}, 1+ r_j)} (x-r_1)^{-\frac{1}{2}} dm \\
 &=\lim\limits_{n \rightarrow \infty} \mathlarger{\int\limits}_{r_j+ \frac{1}{n}}^{1+ r_j}(x-r_1)^{-\frac{1}{2}} dx\\
&=  \mathlarger{\int\limits}_{r_j}^{1+ r_j}(x-r_1)^{-\frac{1}{2}} dx\\
&=2\\
\end{array}
\]
Thus, $f_j \in L^1$ and $2^{-j} \in L^1$ and so by theorem 2.25, $g(x)=\sum_{n=1}^\infty 2^{-n}f(x-r_n)=\sum_{n=1}^\infty 2^{-n}f_n \in L^1$. Also, by theorem 2.25,
\[
\mathlarger{\int}gdm = \mathlarger{\int\limits}_{(r_j, r_j+1)}\sum_{j=1}^\infty 2^{-j}f_j dm = \sum_{j=1}^\infty 2^{-j}\mathlarger{\int\limits}_{(r_j, r_j+1)}f_j dm=\sum_{j=1}^\infty 2^{-j}2=2. \text{ Thus, } g\in L^1. 
\]
\end{pf}
\begin{pf}
Next, we will show $g^2 \not\in L^1$. First, notice \[
g^2 \geq \sum_{n=1}^\infty2^{-2n}(f(x-r_n))^2 \text{ and } (f(x-r_n))^2 = \left\{ \begin{array}{ll}
 \frac{1}{x-r_n} & \text{ if } 0 < x < 1 \\
 0 & \text{ otherwise } 
 \end{array}\right. .
\]	
On $(r_n,1+r_n)$, $\frac{1}{x-r_n}<1$, so, by theorem 2.28, 
\[
\mathlarger{\int\limits}_{(r_n,1+r_n)}\frac{dm}{x-r_n}=\mathlarger{\int\limits}_{r_n}^{1+r_n}\frac{dx}{x-r_n}=\mathlarger{\int\limits}_{0}^{1}\frac{dx}{x}= \infty
\]
Thus, $g^2 \not \in L^1$.
\end{pf}


\end{enumerate}
